%
% PREFACE
%

\chapter*{Preface} 

This library was created when I was studying my degree to ease my academic tasks when I was not forced to use Mathematica, Matlab, SPSS, Linpro, ANSYS, Statgraphics, \dots\\

\BI \ stands for \textbf{BI}bliotec\textbf{A} de pro\textbf{GR}amaci\'on cient\'{\i}fic\textbf{A} which means Scientific Programming Library.\\

About the name? Well, for those days a new medicine was introduced into the market, my mind \dots \\\

First version was ``\emph{published}'' in $1998$ but it is was not widely distributed.\\

Some months ago I started reviewing my backups to centralize useful stuff into only one repository and I found this library. Unfortunately the version I found was one of the first versions so a lot of stuff is missing:
%
\begin{itemize}
\item Matrix operations.
\item Matrix factorization.
\item System linear equations.
\item Partial differential equations.
\item \dots
\end{itemize}
%
This library was originally written in Spanish so I decided to translate into English and making it public available on Github.\\

It is important to remark that this release candidate version has not been tested and the results maybe not accurate enough. I will check and add tests to check it in a not distant future.\\

Today there are a lot of resources in many programming languages but in the days this library was published there was no much stuff available due to access to the internet was not as generalized as nowadays.\\

If you want to know how to use C language for scientific programming this code could be useful, specially how pointers are used to optimize memory usage and how function's pointers are used to abstact C functions from the mathematical functions they are going to use.\\

You can also check how to use threads using \textbf{OpenMP} and how to use the \textbf{GNU Multiple Precision Arithmetic Libary (GMP)}.\\

I will also add some code I created to test how a Fujitsu Primergy server scaled using Pi digits calculations.\\

I do not have any plans to increase this library. If I find the missing code I will add it though.\\

It is possible that I add some code from time to time due to the possibility I have to code some numerical methods in the future.\\

I hope this code is useful for you.\\ \\

\begin{flushright}
Jos\'e \'Angel de Bustos P\'erez  
\end{flushright}
