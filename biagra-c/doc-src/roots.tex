%
% roots.h
%

\chapter{Roots approximation (roots.h)} \label{sec:roots}

\section{Introduction}

Functions to compute function's roots approximation are defined in \texttt{roots.h} file.

\section{Data structures}

Some data structures are defined in \BI to manage roots.

\subsection{\texttt{biaRealRoot} data structure} \label{sec:biaRealRoot}

This data structure is used to store data for root approximation.\\

Data structure is defined in figure \ref{fig:biaRealRoot} where:
%
\begin{description}
\item[intNMI] maximum number of iterations to get the root with a maximum error of \texttt{dblTol}.
\item[intIte] iterations used to get the root.
\item[dblx0] initial approximation to get the root.
\item[dblRoot] root approximation.
\item[dblTol] maximum tolerance when calculating the root.
\item[dblError] error in root approximation. Difference between the las two root approximations.
\end{description}
%
\begin{figure}[!h]
\begin{verbatim}
typedef struct {
  int intNMI,
      intIte;

  double dblx0,
         dblRoot,
         dblTol,
         dblError;
  } biaRealRoot;
\end{verbatim}
\caption{biaRealRoot data structure.} \label{fig:biaRealRoot}
\end{figure}

\FloatBarrier

\section{Function roots approximation}

\subsection{\texttt{newtonPol} function}

This function approaches a polynomial root using the \textbf{Newton} method.\\ \\
%
The definition of this function:
%
\begin{verbatim}
int newtonPol(biaPol *ptPol, biaRealRoot *ptRoot);  
\end{verbatim}
%
The following codes are returned:
%
\begin{center}
\begin{tabular}{|l|l|}
\hline
\textbf{BIA\_MEM\_ALLOC} & Memory allocation error \\
\hline
\textbf{BIA\_ZERO\_DIV} & Division by zero \\
\hline
\textbf{BIA\_TRUE} & the root was computed satisfying the problem \\
                   & conditions (\texttt{intMNI} and \texttt{dblTol}); \\
\hline
\textbf{BIA\_FALSE} & Root approximation could not be calculated \\
                    & satisfying the requirements (\texttt{intMNI} and \texttt{dblTol}). \\
\hline
\end{tabular}
\end{center}
%
The following values in \texttt{*ptRoot} need to be initialized:
%
\begin{description}
\item[intMNI] maximun number of iterations to compute the root.
\item[dblx0] initial approximation.
\item[dblTol] tolerance to compute de root.
\end{description}
%
The following data will be stored:
%
\begin{description}
\item[intIte] iterations used to compute the root.
\item[dblRoot] approximation of the root.
\item[dblError] error in the approximation.
\end{description}
%
\note{When two consecutive approximations are close enough, \texttt{dblTol}, last approximation will be considered as good and will be stored in \texttt{*biaRealRoot *ptRoot} in \texttt{dblRoot}.}
%

\subsection{\texttt{newtonMethod} function}

This function approaches a function's root using the \textbf{Newton} method.\\ \\
%
The definition of this function:
%
\begin{verbatim}
int newtonMethod(biaRealRoot *ptRoot, 
       int (*func)(double dblx0, double *ptRes),
       int (*der)(double dblx0, double *ptRes));  
\end{verbatim}
%
Function's pointers are used to avoid having to recode the C function every time a root function need to be approximated for different mathematical functions.\\ \\
%
The following codes are returned:
%
\begin{center}
\begin{tabular}{|l|l|}
\hline
\textbf{BIA\_ZERO\_DIV} & Division by zero \\
\hline
\textbf{BIA\_TRUE} & the root was computed satisfying the problem \\
                   & conditions (\texttt{intMNI} and \texttt{dblTol}); \\
\hline
\textbf{BIA\_FALSE} & Root approximation could not be calculated \\
                    & satisfying the requirements (\texttt{intMNI} and \texttt{dblTol}). \\
\hline
\end{tabular}
\end{center}
%
where:
%
\begin{description}
\item[ptRoot] is a pointer to a \texttt{biaRealRoot} variable.
\item[func] pointer to a function implementing the function which root is going to be computed.
\item[der] pointer to a function implementing the derivative of the function which root is going to be computed.
\end{description}
%
The following values in \texttt{*ptRoot} need to be initialized:
%
\begin{description}
\item[intMNI] maximun number of iterations to compute the root.
\item[dblx0] initial approximation.
\item[dblTol] tolerance to compute de root.
\end{description}
%
The following data is stored:
%
\begin{description}
\item[intIte] iterations used to compute the root.
\item[dblRoot] approximation of the root.
\item[dblError] error in the approximation.
\end{description}
%
\note{When two consecutive approximations are close enough, \texttt{dblTol}, last approximation will be considered as good and will be stored in \texttt{*biaRealRoot *ptRoot} in \texttt{dblRoot}.}

\subsubsection{Usage example}

To approximate a root for the function $f(x) = \sqrt{2}$ to C functions need to be created. A C function for the mathematical function implementation:
%
\begin{verbatim}
/* f(x) = x^2 - 2 */
int myfunc(double x0, double *fx0) {
  *fx0 = (double)(x0 * x0 - 2.);
  return BIA_TRUE;
}  
\end{verbatim}
%
A C function for the mathematical derivative function implementation:
%
\begin{verbatim}
/* f'(x) = 2*x */
int myfuncder(double x0, double *fx0) {
  *fx0 = 2.*x0;
  return BIA_TRUE;
}  
\end{verbatim}
%
Both functions must meet the following requirements:
%
\begin{itemize}
\item An integer value is returned, \textbf{BIA\_TRUE} when function is evaluated in $x0$ and \textbf{BIA\_ZERO\_DIV} if a division by zero takes place.
\item $x0$ value to evaluate the function.
\item $*fx0$ pointer to a double to store the function's value in $x0$.
\end{itemize}
%
So to approximate function's root using Newton Method:
%
\begin{verbatim}
i = newtonMethod(&myRoot, &myfunc, &myfuncder);  
\end{verbatim}
