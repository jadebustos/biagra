%
% What is B.I.A.G.R.A?
%

\chapter{What is \BI?}

\begin{itemize}

\item \BI stands for \textbf{BI}\emph{bliotec}\textbf{A} \emph{de pro}\textbf{GR}\emph{amaci\'on cient\'{\i}fic}\textbf{A} which means Scientific Programming Library.
\item \BI is enterely coded using \textbf{C} language.
\item \BI has been developed and tested under \emph{LiNUX}.
\item \BI is distribuded as open source and its author does not take any responibility.
\item I wrote \BI in the 90s to help me with some academic tasks when studying my degree.

\end{itemize}

\section{C language?}

\textbf{C} language instead of \textbf{FORTRAN}?
%
\begin{itemize}
\item \textbf{C} is modular and structured.
\item \textbf{C} is a general purpose language programming.
\item \textbf{C} is a very powerful language and its code is very fast.
\item \textbf{C} allows dynamic memory allocation.
\item \textbf{C} code is portable.
\item \textbf{C} is able to handle graphic modes. 
\end{itemize}

\section{Some general ideas about \BI}

\BI\ has been developed under \textbf{Linux} and some \textbf{Linux} knowledge is needed.\\

\BI\ was developed to solve general problems instead of particular ones. For instance, instead of writing a program to get the inverse of a 4x4 matrix and having to change the source code to get the inverse of a 5x5 matrix \BI\ was developed to allow to write programs to get the inverse of any matrix without having to change de sorce code.\\

To be able to do that \emph{pointers} were used instead of using \emph{arrays}.\\

When we talk about \emph{vectors} we will be talking about a \emph{pointer} using dynamic memory allocation. When we talk about matrices we will be talking about \emph{pointer} to a \emph{pointer} using dynamic memory allocation.\\

\BI\ uses some data structures to store data.\\

For common errors as:

\begin{itemize}
\item Errors in dynamic memory allocation.
\item Division by zero.
\item \ldots
\end{itemize}

\BI\ uses its own constants to notify these errors (Chapter \ref{ch:mathematicalConsts}).\\

\section{How to install \BI under LiNUX}

Para instalar \BI lo primero que hay que hacer es entrar en el sistema
como \textbf{root} y situarse en el directorio donde esten los fuentes
de la biblioteca.

\subsection{Intalaci\'on de la biblioteca \BI est\'atica}
Para instalar este tipo bibilioteca se puede hacer de dos formas:

\begin{enumerate}
\item \emph{./instalar estatica}
\item \emph{make estatica}
\end{enumerate}

En realidad ambas hacen lo mismo, la opci\'on con \emph{make} en realidad 
ejecuta \emph{./instalar estatica}.\newline

Al realizar cualquiera de estas dos opciones se copiar\'an los ficheros
cabezera a \emph{\textbf{/usr/include/biagra}} y a continuaci\'on se crear\'a la
biblioteca\\ \emph{\textbf{/usr/lib/libbiagra.a}}.\newline

Luego si se utiliza una funci\'on de esta biblioteca, cuyo prototipo est\'a en 
el fichero de cabecera \emph{rngkutta.h} habr\'a que incluir en las directivas 
al prepocesador \textbf{\#include $<$biagra/rngkutta.h$>$}.
