%
% matrix.h
%

\chapter{Matrix (matrix.h)}

\section{Introduction}

Functions to manage matrices are defined in \texttt{matrix.h} file.\\

\section{Data structures}

Some data structures are defined in \BI \ to manage matrices.

\subsection{\texttt{biaMatrix} data structure} \label{sec:biaMatrix}

This data structure is used to store a matrix. \textbf{biaMatrix} data structure is defined in figure \ref{fig:biaMatrix} where:

\begin{description}
\item[intRows] number of rows.
\item[intCols] number of columns.
\item[dblCoefs] pointer to store matrix coeficients.
\end{description}

\begin{figure}[!h]
\begin{verbatim}
typedef struct {
  int intRows,
      intCols;

  double **dblCoefs;
  } biaMatrix;   
\end{verbatim}
\caption{biaMatrix data structure.} \label{fig:biaMatrix}
\end{figure}

\FloatBarrier

\section{Matrix creation}

\BI \ includes functions to create some kind of matrices.

\subsection{\texttt{identityMatrix} function}

This function stores the identity matrix with order taken from intRows member of \texttt{ptMatrix}:
%
\begin{displaymath}
  \left( \begin{array}{ccccc}
    1 & 0 & 0 & 0 & 0 \\
    0 & 1 & \ddots & 0 & 0 \\
    0 & \ddots & \ddots & \ddots & 0 \\
    0 & 0 & \ddots & 1 & 0 \\
    0 & 0 & 0 & 0 & 1 \\
  \end{array} \right)
\end{displaymath}
%
The definition of this function:
%
\begin{verbatim}
void identityMatrix(biaMatrix *ptMatrix);  
\end{verbatim}
%
where:
%
\begin{description}
\item[*ptMatrix] matrix that has to be created before using this function. Memory allocation for \texttt{dblCoefs} must be done before using this function.
\end{description}
%
\ \\
%
\note{\texttt{intRows} is used to get the matrix order.}

\subsection{\texttt{scalingMatrix} function}

This function stores the scaling matrix with factor $\lambda$ and order taken from intRows member of \texttt{ptMatrix}:
%
\begin{displaymath}
  \left( \begin{array}{ccccc}
    \lambda & 0 & 0 & 0 & 0 \\
    0 & \lambda & \ddots & 0 & 0 \\
    0 & \ddots & \ddots & \ddots & 0 \\
    0 & 0 & \ddots & \lambda & 0 \\
    0 & 0 & 0 & 0 & \lambda \\
  \end{array} \right)
\end{displaymath}
%
The definition of this function:
%
\begin{verbatim}
void scalingMatrix(biaMatrix *ptMatrix, double dblFactor);  
\end{verbatim}
%
where:
%
\begin{description}
\item[*ptMatrix] matrix that has to be created before using this function. Memory allocation for \texttt{dblCoefs} must be done before using this function.
\end{description}
%
\ \\
%
\note{\texttt{intRows} is used to get the matrix order.}

\subsection{\texttt{nullMatrix} function}

This function stores the null matrix with order taken from intRows member of \texttt{ptMatrix}:
%
\begin{displaymath}
  \left( \begin{array}{ccccc}
    0 & 0 & 0 & 0 & 0 \\
    0 & 0 & \ddots & 0 & 0 \\
    0 & \ddots & \ddots & \ddots & 0 \\
    0 & 0 & \ddots & 0 & 0 \\
    0 & 0 & 0 & 0 & 0 \\
  \end{array} \right)
\end{displaymath}
%
The definition of this function:
%
\begin{verbatim}
void nullMatrix(biaMatrix *ptMatrix);  
\end{verbatim}
%
where:
%
\begin{description}
\item[*ptMatrix] matrix that has to be created before using this function. Memory allocation for \texttt{dblCoefs} must be done before using this function.
\end{description}
%
\ \\
%
\note{\texttt{intRows} and \texttt{intCols} is used to get the matrix order.}

\section{Matrix operations}

\subsection{\texttt{transposeMatrix} function}

This function stores the transpose matrix of a given matrix.\\ \\
%
The definition of this function:
%
\begin{verbatim}
void transposeMatrix(biaMatrix *ptMatrix, biaMatrix *ptRes);  
\end{verbatim}
%
where:
%
\begin{description}
\item[*ptMatrix] matrix to get its transpose matrix.
\item[*ptRes] matrix to store the transpose matrix. Memory has to be preallocated before using this function.
\end{description}
%
\ \\
%
\note{\texttt{intRows} and \texttt{intCols} is used to get the matrix order.}

\section{Matrix checks}

\subsection{\texttt{isIdentityMatrix} function}

This function checks if a matrix is the identity matrix.\\ \\
%
The definition of this function:
%
\begin{verbatim}
int isIdentityMatrix(biaMatrix *ptMatrix);
\end{verbatim}
%
where:
%
\begin{description}
\item[*ptMatrix] matrix to check.
\end{description}
%

\subsection{\texttt{isNullMatrix} function}

This function checks if a matrix is a null matrix.\\ \\
%
The definition of this function:
%
\begin{verbatim}
int isNullMatrix(biaMatrix *ptMatrix, double dblTol);
\end{verbatim}
%
where:
%
\begin{description}
\item[*ptMatrix] matrix to check.
\item[dblTol] if a matrix element is minor than this value it is assumed it is a null element.
\end{description}
%

\subsection{\texttt{isSymmetricMatrix} function}

This function checks if a matrix is a symmetric matrix.\\ \\
%
The definition of this function:
%
\begin{verbatim}
int isSymmetricMatrix(biaMatrix *ptMatrix);
\end{verbatim}
%
where:
%
\begin{description}
\item[*ptMatrix] matrix to check.
\end{description}
%
