%
% complex.h
%

\chapter{Complex numbers (complex.h)}

\section{Introduction}

Functions to manage complex numbers are defined in \texttt{complex.h} file.

\section{Data structures}

Some data structures are defined in \BI \ to manage complex numbers.

\subsection{\texttt{biaComplex} data structure} \label{sec:biaComplex}

\textbf{biaComplex} data structure is defined in figure \ref{fig:biaRealPol} where:

\begin{description}
\item[intDegree] polynomial degree.
\item[intRealRoots] number of real roots (if any).
\item[intCompRoots] number of complex roots (if any).
\item[dblCoef] pointer to store polynomial coeficients.
\end{description}

\begin{figure}[!h]
\begin{verbatim}
typedef struct {
  double dblReal,
         dblImag;
  } biaComplex;
\end{verbatim}
\caption{biaComplex data structure.} \label{fig:biaComplex}
\end{figure}

\subsection{\texttt{biaPolar} data structure} \label{sec:biaPolar}

This data structure is used to store data for root approximation. Data structure is defined in figure \ref{fig:biaRealRoot} where:

\begin{description}
\item[intNMI] maximum number of iterations to get the root with a maximum error of \emph{dblTol}.
\item[intIte] iterations used to get the root.
\item[dblx0] initial approximation to get the root.
\item[dblRoot] root approximation.
\item[dblTol] maximum tolerance when calculating the root.
\item[dblError] error in root approximation. Difference between the las two root approximations.
\end{description}

\begin{figure}[!h]
\begin{verbatim}
typedef struct {
  double dblMod,
         dblArg;
  } biaPolar;
\end{verbatim}
\caption{biaPolar data structure.} \label{fig:biaPolar}
\end{figure}

\FloatBarrier

\section{Arithmetical operations using complex numbers}

\subsection{\texttt{addComplex} function}

This function adds two complex numbers.\\

The definition of this function:
%
\begin{verbatim}
void addComplex(biaComplex *ptCmplx1, biaComplex *ptCmplx2, 
                biaComplex *ptRes);
\end{verbatim}
%
where:
%
\begin{description}
\item[*ptCmplx1] first complex number to be added.
\item[*ptCmplx2] second complex number to be added.
\item[*ptRes] result of the operation.
\end{description}

\subsection{\texttt{subtractComplex} function}

This function subtracts two complex numbers.\\

The definition of this function:
%
\begin{verbatim}
void subtractComplex(biaComplex *ptCmplx1, biaComplex *ptCmplx2, 
                     biaComplex *ptRes);  
\end{verbatim}
%
where:
%
\begin{description}
\item[*ptCmplx1] complex number.
\item[*ptCmplx2] complex number to be subtracted to the above.
\item[*ptRes] result of the operation.
\end{description}

\subsection{\texttt{multiplyComplex} function}

This function multiplies two complex numbers.\\

The definition of this function:
%
\begin{verbatim}
void multiplyComplex(biaComplex *ptCmplx1, biaComplex *ptCmplx2, 
                     biaComplex *ptRes);  
\end{verbatim}
%
where:
%
\begin{description}
\item[ptCmplx1] first complex number to be multiplied. 
\item[ptCmplx2] second complex number to be multiplied.
\item[ptRes] result of the operation.
\end{description}

\subsection{\texttt{divideComplex} function}

This function divides one complex number by other:
%
\begin{displaymath}
\frac{\mathrm{a} + \mathrm{b} \cdot i}{\mathrm{c} + \mathrm{d} \cdot i} = (\mathrm{a} + \mathrm{b} \cdot i) \cdot (\mathrm{c} + \mathrm{d} \cdot i)^{-1}  
\end{displaymath}

The definition of this function:
%
\begin{verbatim}
int divideComplex(biaComplex *ptCmplx1, biaComplex *ptCmplx2, 
                  biaComplex *ptRes);
\end{verbatim}
%
where:
%
\begin{description}
\item[*ptCmplx1] complex number.
\item[*ptCmplx2] complex number used as divisor. 
\item[*ptRes] result of the operation.
\end{description}
%
The following codes are returned:
%
\begin{center}
\begin{tabular}{|l|l|}
\hline
\textbf{BIA\_ZERO\_DIV} & Division by zero \\
\hline
\textbf{BIA\_TRUE} & Success \\
\hline
\end{tabular}
\end{center} 

\subsection{\texttt{invSumComplex} function}

This function gets the additive inverse of a complex number:
%
\begin{displaymath}
\forall \ \ z_1 \in \mathbb{C} \quad \exists \ \ z_2 \in \mathbb{C} \ \ | \ \ z_1 + z_2 = 0
\end{displaymath}

The definition of this function:
%
\begin{verbatim}
void invSumComplex(biaComplex *ptCmplx, biaComplex *ptRes);  
\end{verbatim}
%
where:
%
\begin{description}
\item[*ptCmplx] complex number to get its additive inverse.
\item[*ptRes] where the additive inverse will be stored.
\end{description}

\subsection{\texttt{invMulComplex} function}

This function gets the multiplicative inverse of a complex number:
%
\begin{displaymath}
\forall \ \ z_1 \in \mathbb{C} - \{0\} = \mathbb{C}^{*} \quad \exists \ \ z_2 \in \mathbb{C} \ \ | \ \ z_1 \cdot z_2 = 1
\end{displaymath}

The definition of this function:
%
\begin{verbatim}
int invMulComplex(biaComplex *ptCmplx, biaComplex *ptRes) ;  
\end{verbatim}
%
where:
%
\begin{description}
\item[*ptCmplx] complex number to get its multiplicative inverse.
\item[*ptRes] where the additive multiplicative will be stored.
\end{description}

The following codes are returned:

\begin{center}
\begin{tabular}{|l|l|}
\hline
\textbf{BIA\_ZERO\_DIV} & Division by zero \\
\hline
\textbf{BIA\_TRUE} & Success \\
\hline
\end{tabular}
\end{center}

\section{Complex number operations}

\subsection{\texttt{dblComplexModulus} function}

This function gets the modulus of a complex number.\\

The definition of this function:
%
\begin{verbatim}
double dblComplexModule(biaComplex *ptCmplx);  
\end{verbatim}
%
where:
%
\begin{description}
\item[*ptCmplx] complex number to get its modulus.
\end{description}
%
This function returns the complex number modulus.

\subsection{\texttt{dblComplexArg} function}

This function gets the argument of a complex number.\\

The definition of this function:
%
\begin{verbatim}
double dblComplexArg(biaComplex *ptCmplx);  
\end{verbatim}
%
where:
%
\begin{description}
\item[*ptCmplx] complex number to get its argument.
\end{description}
%
This function returns the complex number argument (radians).

\subsection{\texttt{conjugateComplex} function}

This function gets the conjugate complex of a complex number:
%
\begin{displaymath}
z = \textrm{a} + \textrm{b} \cdot i \ \ \in \mathbb{C} \Rightarrow \overline{z} = \textrm{a} - \textrm{b} \cdot i \ \ \in \mathbb{C}
\end{displaymath}
%
The definition of this function:
%
\begin{verbatim}
void conjugateComplex(biaComplex *ptCmplx, biaComplex *ptRes);  
\end{verbatim}
%
where:
%
\begin{description}
\item[*ptCmplx] complex number to get its conjugate.
\item[*ptRes] complex conjugate.
\end{description}

\subsection{\texttt{complex2Polar} function}

This function gets the polar coordinates of a complex number.\\

The definition of this function:
%
\begin{verbatim}
void complex2Polar(biaComplex *ptCmplx, biaPolar *ptRes);  
\end{verbatim}
%
where:
%
\begin{description}
\item[*ptCmplx] complex number to calculate polar coordinates.
\item[*ptRes] polar coordinates.
\end{description}

\subsection{\texttt{polar2Complex} function}

This function gets the cartesian coordinates of a polar coordinates for a complex number.\\

The definition of this function:
%
\begin{verbatim}
void polar2Complex(biaPolar *ptPolar, biaComplex *ptRes);  
\end{verbatim}
%
where:
%
\begin{description}
\item[*ptPolar] polar coordinates.
\item[*ptRes] complex number in cartesian coordinates.
\end{description}

\note{Argument is supposed to be in radians.}
