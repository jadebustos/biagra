%
% RNGKUTTA.H
%

\chapter{Runge-Kutta methods (rngkutta.h)}

\section{Introduction}

\textbf{Runge-Kutta} are a family of implicit and explicit iterative methods used to approximate solutions of ordinary differential equations or \textbf{ODE}.\\

Butcher matricial notation is used in this implementation.\\

\section{Data structures}

\subsection{\texttt{biaButcherArray} data structure} \label{sec:biaButcherArray}

This structure is used to store the Butcher matricial notation.\\

Data structure is defined in figure \ref{fig:biaButcherArray} where:
%
\begin{description}
%
\item[intStages] method stages.
%
\item[*dblC] $c_i$ coefficients stored in an array with size \texttt{intStages}.
%
\item[*dblB] $b_i$ coefficients stored in an array with size \texttt{intStages}.
%
\item[**dblMatrix] matrix to store $a_{i,j}$ method's coeficients.  
%
\end{description}

\begin{figure}[!h]
\begin{verbatim}
typedef struct {
  double  *dblC,
          *dblB,
          **dblMatrix;

  int     intStages;
} biaButcherArray;
\end{verbatim}
\caption{biaButcherArray data structure.} \label{fig:biaButcherArray}
\end{figure}
%
\FloatBarrier

\note{See appendix \ref{ch:runge} if you need information about Butcher matricial notation.}

\subsection{\texttt{biaDataRK} data structure} \label{sec:biaDataRK}

This structure is used to store all the data needed to apply a Runge-Kutta method.\\

Data structure is defined in figure \ref{fig:biaDataRK} where:
%
\begin{description}
%
\item[intNumApprox] number of approximations to be done (size of the array \texttt{dblPoints}).
%
\item[intImplicit] when the Runge-Kutta method is implicit or not. The following constants are defined in the header file:
%
\begin{center}
\begin{tabular}{|c|c|}
\hline
\textbf{Name} & \textbf{Value} \\
\hline
\textbf{BIA\_IMPLICIT\_RK\_TRUE} & $0$ \\
\hline
\textbf{BIA\_IMPLICIT\_RK\_FALSE} & $1$ \\
\hline
\end{tabular}
\end{center}  
%
\item[*dblPoints] array with dimension \texttt{intNumApprox} and its elements will be the approximations in $x_i$ where:
%
\begin{displaymath}
x_i = dblFirst + i \cdot dblStepSize \qquad \textrm{where} \qquad 0 \le i < intNumApprox     
\end{displaymath}
%
\item[dblStepSize] method's step-size.
%
\item[dblFirst] first point used to compute all the approximations. The value of the function in this point is known (initial condition).
%
\item[dblLast] last point in which approximations will be computed.
%
\item[strCoefs] variable of type biaButcherArray (section \ref{sec:biaButcherArray}) storing Butcher matricial notation.
%
\end{description}

\begin{figure}[!h]
\begin{verbatim}
typedef struct {
  int intNumApprox,
      intImplicit;

  double  *dblPoints,
          dblStepSize,
          dblFirst,
          dblLast;

  biaButcherArray strCoefs;
} biaDataRK;
\end{verbatim}
\caption{biaDataRK data structure.} \label{fig:biaDataRK}
\end{figure}
%
\FloatBarrier

\section{Node number calculations}

\subsection{\texttt{intNodeNumber} function}

This function gets the number of nodes that can be placed in an interval. All nodes are equidistant.\\

The definition of this function:
%
\begin{verbatim}
int intNodeNumber(double dblLong, double dblStepSize)
\end{verbatim}
%
where:
%
\begin{description}
\item[dblLong] interval length.
\item[dblStepSize] distance between two nodes.
\end{description}

The function returns the number of nodes that can be placed.\\ \\
%
\note{Arguments are supposed to be different from zero.}
\ \\
\note{Arguments are supposed to be positive.}

\section{Explicit Runge-Kutta methods (scalar problems)}

\subsection{\texttt{explicitRungeKutta} function}

This function solves an \emph{I.V.P.} using an explicit Runge-Kutta method.\\

The definition of this function:
%
\begin{verbatim}
int explicitRungeKutta(biaDataRK *ptData, 
               double (*IVP)(double dblX, double dblY)
\end{verbatim} 
%
where:
%
\begin{description}
\item[ptData] pointer to a \textbf{biaDataRK} variable\footnote{Section (\ref{sec:biaDataRK}).}. This variable contains all the necessary data to solve the \emph{I.V.P.}
\item[IVP] C function's pointer to a function implementing the \emph{O.D.E.}. This functions needs to have two double arguments and returns the value of the \emph{I.V.P.}:
\begin{description}
	\item[dblX] point where we want to evaluate the \emph{O.D.E.}
	\item[dblY] \emph{O.D.E.} value in \texttt{dblX} ($y_i \approx y(x_i)$).
\end{description}
\end{description}

The following codes are returned:
%
\begin{center}
\begin{tabular}{|l|l|}
\hline
\textbf{BIA\_MEM\_ALLOC} & Memory error allocation \\
\hline
\textbf{BIA\_TRUE} & Success \\
\hline
\end{tabular}
\end{center}

\subsubsection{Usage example}

For instance, to solve this \emph{I.V.P.}:
%
\begin{displaymath}
\left \{ \begin{array}{l}
y' = y(x) * \frac{x-y(x)}{x^2} \\
y(1) = 2
\end{array} \right.
\end{displaymath}
%
the implementation of the \texttt{IVP} would be:
%
\begin{verbatim}
double IVP(double dblX, double dblY) {
  double  dblResultado;

  dblRes = dblY*((dblX-dblY)/(dblX*dblX));

  return (dblRes);
}
\end{verbatim}
%
A \texttt{biaDataRK} variable has to be initialized. \BI \ provides several functions to initialize the \texttt{biaButcherArray} member.\\

To solve the \emph{I.V.P.} problem:
%
\begin{verbatim}
i = explicitRungeKutta(&varDataRK, IVP);  
\end{verbatim}

\subsection{\texttt{classicRungeKuttaCoefs} function}

This function initialize the Butcher array for the classic Runge-Kutta method (four-stage method and order two).\\

The definition of this function:
%
\begin{verbatim}
int classicRungeKuttaCoefs(biaDataRK *ptData);
\end{verbatim} 
%
where:
%
\begin{description}
\item[ptData] pointer to a \textbf{biaDataRK} variable\footnote{Section (\ref{sec:biaDataRK}).}. In this variable the following members will be initialized:
%
\begin{description}
%    
\item[intStages] will be initialized to $4$.
%  
\item[intImplicit] will be initialized to \texttt{BIA\_IMPLICIT\_RK\_FALSE}.
%
\item[strCoefs] will be initialized with the Butcher array.
%  
\end{description}
%  
\end{description}
%
The Butcher array for this method:
%
\begin{displaymath}
\begin{array}{c|cccc}
0 & 0 \\
\frac{1}{2} & \frac{1}{2} & 0 \\
\frac{1}{2} & 0 & \frac{1}{2} & 0 \\
1 & 0 & 0 & 1 & 0 \\
\hline
 & \frac{1}{6} & \frac{1}{3} & \frac{1}{3} & \frac{1}{6} \\
\end{array}
\end{displaymath}
%
The following codes are returned:
%
\begin{center}
\begin{tabular}{|l|l|}
\hline
\textbf{BIA\_MEM\_ALLOC} & Memory error allocation \\
\hline
\textbf{BIA\_TRUE} & Success \\
\hline
\end{tabular}
\end{center}

\subsection{\texttt{heunRungeKuttaCoefs} function}

This function initialize the Butcher array for the Heun Runge-Kutta method (three-stage method and order three).\\

The definition of this function:
%
\begin{verbatim}
int heunRungeKuttaCoefs(biaDataRK *ptData);
\end{verbatim} 
%
where:
%
\begin{description}
\item[ptData] pointer to a \textbf{biaDataRK} variable\footnote{Section (\ref{sec:biaDataRK}).}. In this variable the following members will be initialized:
%
\begin{description}
%    
\item[intStages] will be initialized to $3$.
%  
\item[intImplicit] will be initialized to \texttt{BIA\_IMPLICIT\_RK\_FALSE}.
%
\item[strCoefs] will be initialized with the Butcher array.
%  
\end{description}
%  
\end{description}
%
The Butcher array for this method:
%
\begin{displaymath}
\begin{array}{c|ccc}
0 & 0 \\
\frac{1}{3} & \frac{1}{3} & 0 \\
\frac{2}{3} & 0 & \frac{2}{3} & 0 \\
\hline
 & \frac{1}{4} & 0 & \frac{3}{4}
\end{array}
\end{displaymath}
%
The following codes are returned:
%
\begin{center}
\begin{tabular}{|l|l|}
\hline
\textbf{BIA\_MEM\_ALLOC} & Memory error allocation \\
\hline
\textbf{BIA\_TRUE} & Success \\
\hline
\end{tabular}
\end{center}

\subsection{\texttt{kuttaRungeKuttaCoefs} function}

This function initialize the Butcher array for the Kutta Runge-Kutta method (three-stage method and order three).\\

The definition of this function:
%
\begin{verbatim}
int kuttaRungeKuttaCoefs(biaDataRK *ptData);
\end{verbatim} 
%
where:
%
\begin{description}
\item[ptData] pointer to a \textbf{biaDataRK} variable\footnote{Section (\ref{sec:biaDataRK}).}. In this variable the following members will be initialized:
%
\begin{description}
%    
\item[intStages] will be initialized to $3$.
%  
\item[intImplicit] will be initialized to \texttt{BIA\_IMPLICIT\_RK\_FALSE}.
%
\item[strCoefs] will be initialized with the Butcher array.
%  
\end{description}
%  
\end{description}
%
The Butcher array for this method:
%
\begin{displaymath}
\begin{array}{c|ccc}
0 & 0 \\
\frac{1}{2} & \frac{1}{2} & 0 \\
1 & -1 & 2 & 0 \\
\hline
 & \frac{1}{6} & \frac{2}{3} & \frac{1}{6}
\end{array}
\end{displaymath}
%
The following codes are returned:
%
\begin{center}
\begin{tabular}{|l|l|}
\hline
\textbf{BIA\_MEM\_ALLOC} & Memory error allocation \\
\hline
\textbf{BIA\_TRUE} & Success \\
\hline
\end{tabular}
\end{center}

\subsection{\texttt{modifiedEulerRungeKuttaCoefs} function}

This function initialize the Butcher array for the modified Euler Kutta Runge-Kutta method (two-stage method and order two).\\

The definition of this function:
%
\begin{verbatim}
int modifiedEulerRungeKuttaCoefs(biaDataRK *ptData);
\end{verbatim} 
%
where:
%
\begin{description}
\item[ptData] pointer to a \textbf{biaDataRK} variable\footnote{Section (\ref{sec:biaDataRK}).}. In this variable the following members will be initialized:
%
\begin{description}
%    
\item[intStages] will be initialized to $2$.
%  
\item[intImplicit] will be initialized to \texttt{BIA\_IMPLICIT\_RK\_FALSE}.
%
\item[strCoefs] will be initialized with the Butcher array.
%  
\end{description}
%  
\end{description}
%
The Butcher array for this method:
%
\begin{displaymath}
\begin{array}{c|cc}
0 & 0 \\
\frac{1}{2} & \frac{1}{2} & 0 \\
\hline
 & 0 & 1
\end{array}
\end{displaymath}
%
The following codes are returned:
%
\begin{center}
\begin{tabular}{|l|l|}
\hline
\textbf{BIA\_MEM\_ALLOC} & Memory error allocation \\
\hline
\textbf{BIA\_TRUE} & Success \\
\hline
\end{tabular}
\end{center}

\subsection{\texttt{improvedEulerRungeKuttaCoefs} function}

This function initialize the Butcher array for the improved Euler Runge-Kutta method (two-stage method and order two).\\

The definition of this function:
%
\begin{verbatim}
int improvedRungeKuttaCoefs(biaDataRK *ptData);
\end{verbatim} 
%
where:
%
\begin{description}
\item[ptData] pointer to a \textbf{biaDataRK} variable\footnote{Section (\ref{sec:biaDataRK}).}. In this variable the following members will be initialized:
%
\begin{description}
%    
\item[intStages] will be initialized to $3$.
%  
\item[intImplicit] will be initialized to \texttt{BIA\_IMPLICIT\_RK\_FALSE}.
%
\item[strCoefs] will be initialized with the Butcher array.
%  
\end{description}
%  
\end{description}
%
The Butcher array for this method:
%
\begin{displaymath}
\begin{array}{c|cc}
0 & 0 \\
1 & 1 & 0 \\
\hline
 & \frac{1}{2} & \frac{1}{2}
\end{array}
\end{displaymath}
%
The following codes are returned:
%
\begin{center}
\begin{tabular}{|l|l|}
\hline
\textbf{BIA\_MEM\_ALLOC} & Memory error allocation \\
\hline
\textbf{BIA\_TRUE} & Success \\
\hline
\end{tabular}
\end{center}
