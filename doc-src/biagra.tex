\documentclass[a4paper,12pt,twoside,openright]{report}

%\addtolength{\voffset}{-4cm}
\usepackage[english]{babel}
\usepackage[colorlinks = true,
            linkcolor = blue,
            urlcolor  = blue,
            citecolor = blue,
            anchorcolor = blue]{hyperref}
\usepackage{amssymb}
\usepackage{draftwatermark}
\usepackage{placeins}
\usepackage{balloons}
\usepackage{fancyhdr}
\usepackage{hyperref}

%\pagestyle{headings}

\fancyhf{}
\pagestyle{fancy}
\fancyhead [L] {Scientific programming library} %Encabezado
\fancyhead [R]{Version \ver}
\fancyfoot[R]{\thepage} %Pie de página
\fancyfoot[L]{\BI}


\SetWatermarkText{Draft}
\SetWatermarkScale{4}
\SetWatermarkColor{blue}

\author{Jos\'e Angel de Bustos P\'erez}

\newcommand{\BI}{\emph{B.I.A.G.R.A} }
\newcommand{\ver}{RC 1.0}
\newcommand{\me}
	{\emph{Jos\'e Angel de Bustos P\'erez}}

\frenchspacing

\hyphenation{si-guien-te apro-xi-ma-cio-nes FORTRAN double intGrado 
dblDerivada dblResultado intNMI intOrden dblMatriz Newton intResultado
cabezera coor-de-na-das de-pen-dien-do di-men-sio-na-do re-pre-sen-tan-do par-ti-cu-lar a-ppro-xi-ma-tion me-mo-ry po-ly-no-mials e-ve-ry}

\begin{document}

\thispagestyle{empty}

\textbf{
\begin{center}
\Huge{B.I.A.G.R.A.} \\[.75cm]
%\end{center}
\LARGE BI\Large bliotec\LARGE A \Large de pro\LARGE GR\Large amaci\'on 
cient\'{\i}fic\LARGE A\\[5cm]
\end{center}
}

\large 
\begin{flushright}
\me \\
$<$jadebustos@gmail.com$>$\\ \ \\ 
Version \ver, \today .\\
\textbf{\LaTeXe}
\end{flushright}

\normalsize

\tableofcontents

%
% PREFACE
%

\chapter*{Preface} 

This library was created when I was studying my degree to ease my academic tasks when I was not forced to use Mathematica, Matlab, SPSS, Linpro, ANSYS, Statgraphics, \dots\\

\BI \ stands for \textbf{BI}bliotec\textbf{A} de pro\textbf{GR}amaci\'on cient\'{\i}fic\textbf{A} which means Scientific Programming Library.\\

About the name? Well, for those days a new medicine was introduced into the market, my mind \dots \\\

First version was ``\emph{published}'' in $1998$ but it is was not widely distributed.\\

Some months ago I started reviewing my backups to centralize useful stuff into only one repository and I found this library. Unfortunately the version I found was one of the first versions so a lot of stuff is missing:
%
\begin{itemize}
\item Matrix operations.
\item Matrix factorization.
\item System linear equations.
\item Partial differential equations.
\item \dots
\end{itemize}
%
This library was originally written in Spanish so I decided to translate into English and making it public available on Github.\\

It is important to remark that this release candidate version has not been tested and the results maybe not accurate enough. I will check and add tests to check it in a not distant future.\\

Today there are a lot of resources in many programming languages but in the days this library was published there was no much stuff available due to access to the internet was not as generalized as nowadays.\\

If you want to know how to use C language for scientific programming this code could be useful, specially how pointers are used to optimize memory usage and how function's pointers are used to abstact C functions from the mathematical functions they are going to use.\\

You can also check how to use threads using \textbf{OpenMP} and how to use the \textbf{GNU Multiple Precision Arithmetic Libary (GMP)}.\\

I will also add some code I created to test how a Fujitsu Primergy server scaled using Pi digits calculations.\\

I do not have any plans to increase this library. If I find the missing code I will add it though.\\

It is possible that I add some code from time to time due to the possibility I have to code some numerical methods in the future.\\

I hope this code is useful for you.\\ \\

\begin{flushright}
Jos\'e \'Angel de Bustos P\'erez  
\end{flushright}


\part{Introduction}

%
% INTRODUCTION
%

%
% What is B.I.A.G.R.A?
%

\chapter{What is \BI?}

\begin{itemize}

\item \BI stands for \textbf{BI}\emph{bliotec}\textbf{A} \emph{de pro}\textbf{GR}\emph{amaci\'on cient\'{\i}fic}\textbf{A} which means Scientific Programming Library.
\item \BI is enterely coded using \textbf{C} language.
\item \BI has been developed and tested under \emph{LiNUX}.
\item \BI is distribuded as open source and its author does not take any responibility.
\item I wrote \BI in the 90s to help me with some academic tasks when studying my degree.

\end{itemize}

\section{C language?}

\textbf{C} language instead of \textbf{FORTRAN}?
%
\begin{itemize}
\item \textbf{C} is modular and structured.
\item \textbf{C} is a general purpose language programming.
\item \textbf{C} is a very powerful language and its code is very fast.
\item \textbf{C} allows dynamic memory allocation.
\item \textbf{C} code is portable.
\item \textbf{C} is able to handle graphic modes. 
\end{itemize}

\section{Some general ideas about \BI}

\BI\ has been developed under \textbf{Linux} and some \textbf{Linux} knowledge is needed.\\

\BI\ was developed to solve general problems instead of particular ones. For instance, instead of writing a program to get the inverse of a 4x4 matrix and having to change the source code to get the inverse of a 5x5 matrix \BI\ was developed to allow to write programs to get the inverse of any matrix without having to change de sorce code.\\

To be able to do that \emph{pointers} were used instead of using \emph{arrays}.\\

When we talk about \emph{vectors} we will be talking about a \emph{pointer} using dynamic memory allocation. When we talk about matrices we will be talking about \emph{pointer} to a \emph{pointer} using dynamic memory allocation.\\

\BI\ uses some data structures to store data.\\

For common errors as:

\begin{itemize}
\item Errors in dynamic memory allocation.
\item Division by zero.
\item \ldots
\end{itemize}

\BI\ uses its own constants to notify these errors (Chapter \ref{ch:mathematicalConsts}).\\

\section{How to install \BI under LiNUX}

Para instalar \BI lo primero que hay que hacer es entrar en el sistema
como \textbf{root} y situarse en el directorio donde esten los fuentes
de la biblioteca.

\subsection{Intalaci\'on de la biblioteca \BI est\'atica}
Para instalar este tipo bibilioteca se puede hacer de dos formas:

\begin{enumerate}
\item \emph{./instalar estatica}
\item \emph{make estatica}
\end{enumerate}

En realidad ambas hacen lo mismo, la opci\'on con \emph{make} en realidad 
ejecuta \emph{./instalar estatica}.\newline

Al realizar cualquiera de estas dos opciones se copiar\'an los ficheros
cabezera a \emph{\textbf{/usr/include/biagra}} y a continuaci\'on se crear\'a la
biblioteca\\ \emph{\textbf{/usr/lib/libbiagra.a}}.\newline

Luego si se utiliza una funci\'on de esta biblioteca, cuyo prototipo est\'a en 
el fichero de cabecera \emph{rngkutta.h} habr\'a que incluir en las directivas 
al prepocesador \textbf{\#include $<$biagra/rngkutta.h$>$}.


\part{B.I.A.G.R.A Data structures and constants}

%
% const.h
%

%
% const.h
%

\chapter{B.I.A.G.R.A constants (const.h)} \label{ch:mathematicalConsts}

\section{Introduction}

\BI \ includes its own constants to be used if needed.\\

These constants are defined in \texttt{const.h}.

\section{Mathematical constants} \label{sec:mathematicalConsts}

Table \ref{tab:mathematicalConsts} shows the \BI's mathematical constanst.

\begin{table}[!h]
  \begin{center}
  \begin{tabular}{|c|c|c|}
    \hline
    \textbf{Constant} & \textbf{Name} & \textbf{Value} \\
    \hline
    $\texttt{e}$ & \texttt{BIA\_E} & $2.71828182845904523536029$ \\
    \hline
    $\pi$ & \texttt{BIA\_PI} & 3.14159265358979323846264 \\
    \hline
  \end{tabular}
  \end{center}
\caption{\BI\ mathematical constants.} \label{tab:mathematicalConsts}
\end{table}

\FloatBarrier

\section{Logical constants}

The following logical constants are defined:

\begin{description}
\item[BIA\_FALSE] when a condition is not met.
\item[BIA\_TRUE] when a condiction is met.
\end{description}

\section{Error constants}

The following error constants are defined:

\begin{description}
\item[BIA\_ZERO\_DIV] division by zero.
\item[BIA\_MEM\_ALLOC] error in memory allocation.
\end{description}


%
% dataaprox.h
%

\include{dataaprox}

\part{Memory allocation}

%
% resmem.h
%

%
% resmem.h
%

\chapter{Memory allocation (resmem.h)}

\section{Introduction}

\BI\ includes its own memory allocation functions which are defined in \texttt{resmem.h} file.

\section{Vector's memory allocation}

Some functions are provided to handle memory allocations for vectors.

\subsection{\texttt{intPtMemAllocVec} function} \label{sec:intPtMemAllocVec}

This functions allocates memory for a vector of int.\\

The definition of this function:
%
\begin{verbatim}
double *intPtMemAllocVec(int intElements);  
\end{verbatim}
%
This function has only one argument, \texttt{intElements}, which is the dimension of the vector and a \texttt{int} pointer is returned.

\subsection{\texttt{dblPtMemAllocVec} function} \label{sec:dblPtMemAllocVec}

This functions allocates memory for a vector of doubles.\\

The definition of this function:
%
\begin{verbatim}
double *dblPtMemAllocVec(int intElements);  
\end{verbatim}
%
This function has only one argument, \texttt{intElements}, which is the dimension of the vector and a \texttt{double} pointer is returned.

\section{Matrix memory allocation}

Some functions are provided to handle memory allocations for vectors.

\subsection{\texttt{dblPtMemAllocMat} function} \label{sec:dblPtMemAllocMat}

This function allocates memory for a matrix of doubles.\\

The definition of this function:
%
\begin{verbatim}
double **dblPtMemAllocMat(int intRows, int intCols);
\end{verbatim}
%
where:
%
\begin{description}
\item[intRows] number of rows.
\item[intCols] number of columns.
\end{description}

\subsection{\texttt{dblPtMemAllocUpperTrMat} function} \label{sec:dblPtMemAllocUpperTrMat}

This function allocates memory for a upper triangular square matrix.\\

The definition of this function:
%
\begin{verbatim}
double **dblPtMemAllocUpperTrMat(int intOrder);
\end{verbatim}

This function has only one argument, \texttt{intOrder}, which is the order of the matrix and a \texttt{double} pointer to pointer is returned.\\

In a upper triangular square matrix all elements below the diagonal are zero: 
%
\begin{displaymath}
\left( \begin{array}{ccccc}
  a_{0,0} & a_{0,1} & a_{0,2} & a_{0,3} & a_{0,4} \\
  0      & a_{1,1} & a_{1,2} & a_{1,3} & a_{1,4} \\ 
  0      & 0      & a_{2,2} & a_{2,3} & a_{2,4} \\
  0      & 0      & 0      & a_{3,3} & a_{3,4} \\
  0      & 0      & 0      & 0      & a_{4,4} \\
\end{array} \right)
\end{displaymath}
%
For \texttt{intOrder = 5}:
%
\begin{verbatim}
myMatrix = dbpPtMemAllocUpperTrMat(5);  
\end{verbatim}
%
and:
\begin{center}
  \begin{tabular}{|c|c|c|c|}
    \hline
    \textbf{Pointer} & \textbf{\# elements} & \textbf{First element} & \textbf{Last element}\\
    \hline
    \texttt{myMatrix[0]} & 5 & 0 & 4\\
    \hline
    \texttt{myMatrix[1]} & 4 & 0 & 3\\
    \hline
    \texttt{myMatrix[2]} & 3 & 0 & 2\\
    \hline
    \texttt{myMatrix[3]} & 2 & 0 & 1\\
    \hline
    \texttt{myMatrix[4]} & 1 & 0 & 0\\
    \hline
  \end{tabular}
\end{center}
%
so:
%
\begin{displaymath}
  myMatrix[i][j] = *(*(myMatrix + i) + j) = \left\{ \begin{array}{ll}
    a_{i,j+i} & \forall \ i \le j \\
     & \\
    0 & \forall \ i > j
    \end{array} \right.    
\end{displaymath}

\subsection{\texttt{dblPtMemAllocLowerTrMat} function} \label{sec:dblPtMemAllocLowerTrMat}

This function allowcates memory for a lower triangular square matrix.\\

The definition of this function:
%
\begin{verbatim}
double **dblPtMemAllocLowerTrMat(int intOrder);
\end{verbatim}

This function has only one argument, \texttt{intOrder}, which is the order of the matrix and a \texttt{double} pointer to pointer is returned.\\

In a lower triangular square matrix all elements above the diagonal are zero: 
%
\begin{displaymath}
\left( \begin{array}{ccccc}
  a_{0,0} & 0      & 0      & 0      & 0 \\
  a_{1,0} & a_{1,1} & 0      & 0      & 0 \\ 
  a_{2,0} & a_{2,1} & a_{2,2} & 0      & 0 \\
  a_{3,0} & a_{3,1} & a_{3,2} & a_{3,3} & 0 \\
  a_{4,0} & a_{4,1} & a_{4,2} & a_{4,3} & a_{4,4} \\
\end{array} \right)
\end{displaymath}
%
For \texttt{intOrder = 5}:
%
\begin{verbatim}
myMatrix = dbpPtMemAllocLowerTrMat(5);  
\end{verbatim}
%
and:
\begin{center}
  \begin{tabular}{|c|c|c|c|}
    \hline
    \textbf{Pointer} & \textbf{\# elements} & \textbf{First element} & \textbf{Last element}\\
    \hline
    \texttt{myMatrix[0]} & 1 & 0 & 0\\
    \hline
    \texttt{myMatrix[1]} & 2 & 0 & 1\\
    \hline
    \texttt{myMatrix[2]} & 3 & 0 & 2\\
    \hline
    \texttt{myMatrix[3]} & 4 & 0 & 3\\
    \hline
    \texttt{myMatrix[4]} & 5 & 0 & 4\\
    \hline
  \end{tabular}
\end{center}
%
so:
%
\begin{displaymath}
  myMatrix[i][j] = *(*(myMatrix + i) + j) = \left\{ \begin{array}{ll}
    a_{i,j} & \forall \ i \le j \\
     & \\
    0 & \forall \ i < j
    \end{array} \right.    
\end{displaymath}

\section{Freeing memory} \label{sec:freeingMemory}

\BI\ includes its own functions to free memory.

\subsection{\texttt{freeMemDblMat} function} \label{sec:freeMemDblMat}




%
% MATHEMATICAL FUNCTIONS
%

\part{Mathematical functions}

%
% random.h
%

%
% random.h
%

\chapter{Pseudo random numbers (random.h)} \label{ch:random}

\section{Introduction}

\BI \ includes its own functions to pseudo random number generation and they are defined in \texttt{random.h} file.\\

\warning{This functions have not been tested to produce unpredictable sequences, so be careful when use them.}

\section{Pseudo random integer numbers}

\subsection{\texttt{intRandom} function} \label{sec:intRandom}

This function generates random integers.\\

The definition of this function:
%
\begin{verbatim}
int intRandom(int limit);  
\end{verbatim}
%
The pseudo random integer is placed in the interval $(-limit,limit)$.\\ \\
%
\note{Before using this function \texttt{srand} must be used to initialize \texttt{rand}. You can use \texttt{srand((unsigned)time(NULL))}.}
%
\ \\
%
The pseudo random number is generated with the following formula:
%
\begin{displaymath}
\left[ \frac{limit \cdot rand()}{RAND\_MAX + 1} \right] \in (-limit,limit)
\end{displaymath}
%
Then randomly is choosed if the number is positive or negative using the above formula with $limit=2$ and then taking modulus $2$. If modulus is $1$ then the number will be a negative one.

\subsection{\texttt{uintRandom} function} \label{sec:uintRandom}

This function generates random integers.\\

The definition of this function:
%
\begin{verbatim}
int uintRandom(int limit);  
\end{verbatim}
%
The pseudo random integer is placed in the interval $[0,limit)$.\\ \\
%
\note{Before using this function \texttt{srand} must be used to initialize \texttt{rand}. You can use \texttt{srand((unsigned)time(NULL))}.}
%
\ \\
%
The pseudo random number is generated with the following formula:
%
\begin{displaymath}
\left[ \frac{limit \cdot rand()}{RAND\_MAX + 1} \right] \in [0,limit)
\end{displaymath}
%

\section{Pseudo random floating point numbers}

\subsection{\texttt{dblRandom} function} \label{sec:dblRandom}

This function generates random floating point numbers.\\

The definition of this function:
%
\begin{verbatim}
int dblRandom(int limit);  
\end{verbatim}
%
The pseudo random floating point number is placed in the interval $(-limit,limit)$.\\ \\
%
\note{Before using this function \texttt{srand} must be used to initialize \texttt{rand}. You can use \texttt{srand((unsigned)time(NULL))}.}
%
\ \\
%
The pseudo random number is generated with the following formula:
%
\begin{displaymath}
\frac{limit \cdot rand()}{RAND\_MAX + 1} \in (-limit,limit)
\end{displaymath}
%
Then randomly is choosed if the number is positive or negative using the above formula with $limit=2$ and then taking modulus $2$. If modulus is $1$ then the number will be a negative one.

\subsection{\texttt{udblRandom} function} \label{sec:udblRandom}

This function generates random floating point numbers.\\

The definition of this function:
%
\begin{verbatim}
int udblRandom(int limit);  
\end{verbatim}
%
The pseudo random floating point number is placed in the interval $[0,limit)$.\\ \\
%
\note{Before using this function \texttt{srand} must be used to initialize \texttt{rand}. You can use \texttt{srand((unsigned)time(NULL))}.}
%
\ \\
%
The pseudo random number is generated with the following formula:
%
\begin{displaymath}
\frac{limit \cdot rand()}{RAND\_MAX + 1} \in [0,limit)
\end{displaymath}
%


%
% complex.h
%

%
% complex.h
%

\chapter{Complex numbers (complejo.h)}

\section{Introduction}

Functions to manage complex numbers are defined in \texttt{complex.h} file.

\section{Data structures}

Some data structures are defined in \BI \ to manage complex numbers.

\subsection{\texttt{biaComplex} data structure} \label{sec:biaComplex}

This data structure is used to handle polinomials $p(x) \in \mathbb{R}[x]$. \textbf{biaComplex} data structure is defined in figure \ref{fig:biaRealPol} where:

\begin{description}
\item[intDegree] polynomial degree.
\item[intRealRoots] number of real roots (if any).
\item[intCompRoots] number of complex roots (if any).
\item[*dblCoef] pointer to store polynomial coeficients.
\end{description}

\begin{figure}[!h]
\begin{verbatim}
typedef struct {
  double dblReal,
         dblImag;
  } biaComplex;
\end{verbatim}
\caption{biaComplex data structure.} \label{fig:biaComplex}
\end{figure}

\subsection{\texttt{biaPolar} data structure} \label{sec:biaPolar}

This data structure is used to store data for root approximation. Data structure is defined in figure \ref{fig:biaRealRoot} where:

\begin{description}
\item[intNMI] maximum number of iterations to get the root with a maximum error of \emph{dblTol}.
\item[intIte] iterations used to get the root.
\item[dblx0] initial approximation to get the root.
\item[dblRoot] root approximation.
\item[dblTol] maximum tolerance when calculating the root.
\item[dblError] error in root approximation. Difference between the las two root approximations.
\end{description}

\begin{figure}[!h]
\begin{verbatim}
typedef struct {
  double dblMod,
         dblArg;
  } biaPolar;
\end{verbatim}
\caption{biaPolar data structure.} \label{fig:biaPolar}
\end{figure}

\FloatBarrier

\section{Arithmetical operations using complex numbers}

\subsection{\texttt{addComplex} function}

This function adds two complex numbers.\\

The definition of this function:
%
\begin{verbatim}
void addComplex(biaComplex *ptCmplx1, biaComplex *ptCmplx2, 
                biaComplex *ptRes);
\end{verbatim}
%
where:
%
\begin{description}
\item[*ptCmplx1] first complex number to be added.
\item[*ptCmplx2] second complex number to be added.
\item[*ptRes] result of the operation.
\end{description}

\subsection{\texttt{subtractComplex} function}

This function subtracts two complex numbers.\\

The definition of this function:
%
\begin{verbatim}
void subtractComplex(biaComplex *ptCmplx1, biaComplex *ptCmplx2, 
                     biaComplex *ptRes);  
\end{verbatim}
%
where:
%
\begin{description}
\item[*ptCmplx1] complex number.
\item[*ptCmplx2] complex number to be subtracted to the above.
\item[*ptRes] result of the operation.
\end{description}

\subsection{\texttt{multiplyComplex} function}

This function multiplies two complex numbers.\\

The definition of this function:
%
\begin{verbatim}
void multiplyComplex(biaComplex *ptCmplx1, biaComplex *ptCmplx2, 
                     biaComplex *ptRes);  
\end{verbatim}
%
where:
%
\begin{description}
\item[ptCmplx1] first complex number to be multiplied. 
\item[ptCmplx2] second complex number to be multiplied.
\item[ptRes] result of the operation.
\end{description}

\subsection{\texttt{divideComplex} function}

This function divides one complex number by other:
%
\begin{displaymath}
\frac{\mathrm{a} + \mathrm{b} \cdot i}{\mathrm{c} + \mathrm{d} \cdot i} = (\mathrm{a} + \mathrm{b} \cdot i) \cdot (\mathrm{c} + \mathrm{d} \cdot i)^{-1}  
\end{displaymath}

The definition of this function:
%
\begin{verbatim}
int divideComplex(biaComplex *ptCmplx1, biaComplex *ptCmplx2, 
                  biaComplex *ptRes);
\end{verbatim}
%
where:
%
\begin{description}
\item[*ptCmplx1] complex number.
\item[*ptCmplx2] complex number used as divisor. 
\item[*ptRes] result of the operation.
\end{description}
%
The following codes are returned:
%
\begin{center}
\begin{tabular}{|l|l|}
\hline
\textbf{BIA\_ZERO\_DIV} & Division by zero \\
\hline
\textbf{BIA\_TRUE} & Success \\
\hline
\end{tabular}
\end{center} 

\subsection{\texttt{invSumComplex} function}

This function gets the additive inverse of a complex number:
%
\begin{displaymath}
\forall \ \ z_1 \in \mathbb{C} \quad \exists \ \ z_2 \in \mathbb{C} \ \ | \ \ z_1 + z_2 = 0
\end{displaymath}

The definition of this function:
%
\begin{verbatim}
void invSumComplex(biaComplex *ptCmplx, biaComplex *ptRes);  
\end{verbatim}
%
where:
%
\begin{description}
\item[*ptCmplx] complex number to get its additive inverse.
\item[*ptRes] where the additive inverse will be stored.
\end{description}

\subsection{\texttt{invMulComplex} function}

This function gets the multiplicative inverse of a complex number:
%
\begin{displaymath}
\forall \ \ z_1 \in \mathbb{C} - \{0\} = \mathbb{C}^{*} \quad \exists \ \ z_2 \in \mathbb{C} \ \ | \ \ z_1 \cdot z_2 = 1
\end{displaymath}

The definition of this function:
%
\begin{verbatim}
int invMulComplex(biaComplex *ptCmplx, biaComplex *ptRes) ;  
\end{verbatim}
%
where:
%
\begin{description}
\item[*ptCmplx] complex number to get its multiplicative inverse.
\item[*ptRes] where the additive multiplicative will be stored.
\end{description}

The following codes are returned:

\begin{center}
\begin{tabular}{|l|l|}
\hline
\textbf{BIA\_ZERO\_DIV} & Division by zero \\
\hline
\textbf{BIA\_TRUE} & Success \\
\hline
\end{tabular}
\end{center}

\section{Complex number operations}

\subsection{\texttt{dblComplexModulus} function}

This function gets the modulus of a complex number.\\

The definition of this function:
%
\begin{verbatim}
double dblComplexModule(biaComplex *ptCmplx);  
\end{verbatim}
%
where:
%
\begin{description}
\item[*ptCmplx] complex number to get its modulus.
\end{description}
%
This function returns the complex number modulus.

\subsection{\texttt{dblComplexArg} function}

This function gets the argument of a complex number.\\

The definition of this function:
%
\begin{verbatim}
double dblComplexArg(biaComplex *ptCmplx);  
\end{verbatim}
%
where:
%
\begin{description}
\item[*ptCmplx] complex number to get its argument.
\end{description}
%
This function returns the complex number argument (radians).

\subsection{\texttt{conjugateComplex} function}

This function gets the conjugate complex of a complex number:
%
\begin{displaymath}
z = \textrm{a} + \textrm{b} \cdot i \ \ \in \mathbb{C} \Rightarrow \overline{z} = \textrm{a} - \textrm{b} \cdot i \ \ \in \mathbb{C}
\end{displaymath}
%
The definition of this function:
%
\begin{verbatim}
void conjugateComplex(biaComplex *ptCmplx, biaComplex *ptRes);  
\end{verbatim}
%
where:
%
\begin{description}
\item[*ptCmplx] complex number to get its conjugate.
\item[*ptRes] complex conjugate.
\end{description}

\subsection{\texttt{complex2Polar} function}

This function gets the polar coordinates of a complex number.\\

The definition of this function:
%
\begin{verbatim}
void complex2Polar(biaComplex *ptCmplx, biaPolar *ptRes);  
\end{verbatim}
%
where:
%
\begin{description}
\item[*ptCmplx] complex number to calculate polar coordinates.
\item[*ptRes] polar coordinates.
\end{description}

\subsection{\texttt{polar2Complex} function}

This function gets the cartesian coordinates of a polar coordinates for a complex number.\\

The definition of this function:
%
\begin{verbatim}
void polar2Complex(biaPolar *ptPolar, biaComplex *ptRes);  
\end{verbatim}
%
where:
%
\begin{description}
\item[*ptPolar] polar coordinates.
\item[*ptRes] complex number in cartesian coordinates.
\end{description}

\note{Argument is supposed to be in radians.}


%
% integers.h
%

%
% integers.h
%

\chapter{Integer numbers (integers.h)} \label{ch:integers}

\section{Introduction}

\BI \ includes functions about integer numbers in \texttt{integers.h} file.\\

\section{Sum integers}

\subsection{\texttt{uintSumFirstNIntegers} function} \label{sec:uintSumFirstNIntegers}

This function gets the sum of the first $n$ integers.\\ \\
%
The definition of this function:
%
\begin{verbatim}
unsigned uintSumFirstNIntegers(int n);
\end{verbatim}
%
If the sum is bigger than an unsigned int $0$ is returned.

\section{Prime numbers}

\subsection{\texttt{isPrime} function}

This function checks if a number is a prime number.\\ \\
%
The definition of this function:
%
\begin{verbatim}
int isPrime(int intN);  
\end{verbatim}
%
The following codes are returned:

\begin{center}
\begin{tabular}{|l|l|}
\hline
\textbf{BIA\_FALSE} & \texttt{intN} is not a prime number \\
\hline
\textbf{BIA\_TRUE} & \texttt{intN} is a prime number \\
\hline
\end{tabular}
\end{center}

\subsection{\texttt{getFirstNPrimes} function}

This function checks if a number is a prime number.\\ \\
%
The definition of this function:
%
\begin{verbatim}
void getFirstNPrimes(unsigned int *ptPrimes, int intNumber, int *ptCalc);
\end{verbatim}
%
where:
%
\begin{description}
\item[*ptPrimes] array where primes will be stored. Memory allocation for this array has to be initialized before using this function.
\item[intNumber] number of primes to be computed.
\item[*ptCalc] in this variable the total amount of computed primes will be stored.
\end{description}


%
% poliynomial.h
%

%
% polynomial.h
%

\chapter{Polynomial (polynomials.h)} \label{sec:polynomial}

\section{Introduction}

Functions to manage polynomials are defined in \texttt{polynomial.h} file.\\

A polynomial used to be represented as shown in equation \ref{equ:pol}.

\begin{equation} \label{equ:pol}
p(x) = a_0 + a_1 \cdot x + \cdots + a_n \cdot x^n = \sum_{i=0}^n a_i \cdot x^i \qquad \textrm{where } a_i \in \mathbb{R}
\end{equation}

\section{Data structures}

Some data structures are defined in \BI to manage polynomials.

\subsection{\texttt{biaRealPol} data structure} \label{sec:biaRealPol}

This data structure is used to handle polinomials $p(x) \in \mathbb{R}[x]$. \textbf{biaPol} data structure is defined in figure \ref{fig:biaRealPol} where:
%
\begin{description}
\item[intDegree] polynomial degree.
\item[intRealRoots] number of real roots (if any).
\item[intCompRoots] number of complex roots (if any).
\item[dblCoef] pointer to store polynomial coeficients.
\end{description}
%
\begin{figure}[!h]
\begin{verbatim}
typedef struct {
  int  intDegree    = 0,
       intRealRoots = 0,
       intCompRoots = 0;

  double  *dblCoefs;
  } biaRealPol;
\end{verbatim}
\caption{biaRealPol data structure.} \label{fig:biaRealPol}
\end{figure}
%
\FloatBarrier
%
Polynomial coeficients are stored in \texttt{dblCoefs} pointer which has to be previously initialized:
%
\begin{eqnarray*}
  \mathrm{dblCoefs[0]} & = & a_0 \\
  \mathrm{dblCoefs[1]} & = & a_1 \\
  \dots & \dots & \dots \\
  \mathrm{dblCoefs[n]} & = & a_n \\
\end{eqnarray*}

\section{Polynomial derivatives}

\subsection{\texttt{derivativePol} function}

This function gets the $n$-th derivative of a polynomial.\\ \\
%
The definition of this function:
%
\begin{verbatim}
int derivativePol(biaPol *ptPol, biaPol *ptDer, int intN);
\end{verbatim}
%
where:
\begin{description} 
\item[ptPol] pointer to a \texttt{biaRealPol} struct with the polynomial to get its derivative is stored.
\item[ptDer] pointer to a \texttt{biaRealPol} struct where the derivative will be stored.
\item[intN] order of the derivative to get.
\end{description}
%
The following codes are returned:
%
\begin{center}
\begin{tabular}{|l|l|}
\hline
\textbf{BIA\_MEM\_ALLOC} & Memory allocation error \\
\hline
\textbf{BIA\_TRUE} & Success \\
\hline
\end{tabular}
\end{center}
%
\note{\texttt{ptDer} will be released and memory allocation will be carried out to store the derivative.}
%
\warning{\texttt{ptDer} member \texttt{dblCoefs} has to be initialized to a \texttt{NULL} pointer to avoid a \textbf{Segment Fault} error if it was not previously initialized.}

\section{Arithmetical operations using polynomials}

\subsection{\texttt{addPol} function}

This function adds two polynomials.\\ \\
%
The definition of this function:
%
\begin{verbatim}
int addPol(biaPol *ptPol1, biaPol *ptPol2, biaPol *ptRes);  
\end{verbatim}
%
where:
%
\begin{description}
\item[ptPol1] pointer to a \texttt{biaPol} struct with the first polynomial to be added.
\item[ptPol2] pointer to a \texttt{biaPol} struct with the second polynomial to be added.
\item[ptRes] pointer to a \texttt{biaPol} struct where the add operation will be stored.
\end{description}
%
The following codes are returned:
%
\begin{center}
\begin{tabular}{|l|l|}
\hline
\textbf{BIA\_MEM\_ALLOC} & Memory allocation error \\
\hline
\textbf{BIA\_TRUE} & Success \\
\hline
\end{tabular}
\end{center}
%
\note{\texttt{ptRes} will be released and memory allocation will be carried out to store the derivative.}
%
\warning{\texttt{ptRes} member \texttt{dblCoefs} has to be initialized to a \texttt{NULL} pointer to avoid a \textbf{Segment Fault} error if it was not previously initialized.}

\subsection{\texttt{subtractPol} function}

This function subtracts two polynomials.\\ \\
%
The definition of this function:
%
\begin{verbatim}
int subtractPol(biaPol *ptPol1, biaPol *ptPol2, biaPol *ptRes);
\end{verbatim}
%
where:
%
\begin{description}
\item[ptPol1] pointer to a \texttt{biaPol} struct with the first polynomial.
\item[ptPol2] pointer to a \texttt{biaPol} struct with the polynomial to be subtracted from the above.
\item[ptRes] pointer to a \texttt{biaPol} struct where the subtract operation will be stored.
\end{description}
%
The following codes are returned:
%
\begin{center}
\begin{tabular}{|l|l|}
\hline
\textbf{BIA\_MEM\_ALLOC} & Memory allocation error \\
\hline
\textbf{BIA\_TRUE} & Success \\
\hline
\end{tabular}
\end{center}
%
\note{\texttt{ptRes} will be released and memory allocation will be carried out to store the derivative.}
%
\warning{\texttt{ptRes} member \texttt{dblCoefs} has to be initialized to a \texttt{NULL} pointer to avoid a \textbf{Segment Fault} error if it was not previously initialized.}

\subsection{\texttt{multiplyPol} function}

This functions multiplies two polynomials.\\ \\
%
The definition of this function:
%
\begin{verbatim}
int subtractPol(biaPol *ptPol1, biaPol *ptPol2, biaPol *ptRes);
\end{verbatim}
%
where:
%
\begin{description}
\item[ptPol1] pointer to a \texttt{biaPol} struct with the first polynomial.
\item[ptPol2] pointer to a \texttt{biaPol} struct with the second polynomial.
\item[ptRes] pointer to a \texttt{biaPol} struct where the multiplication operation will be stored.
\end{description}
%
The following codes are returned:
%
\begin{center}
\begin{tabular}{|l|l|}
\hline
\textbf{BIA\_MEM\_ALLOC} & Memory allocation error \\
\hline
\textbf{BIA\_TRUE} & Success \\
\hline
\end{tabular}
\end{center}
%
\note{\texttt{ptRes} will be released and memory allocation will be carried out to store the derivative.}
%
\warning{\texttt{ptRes} member \texttt{dblCoefs} has to be initialized to a \texttt{NULL} pointer to avoid a \textbf{Segment Fault} error if it was not previously initialized.}








%
% matrix.h
%

%
% matrix.h
%

\chapter{Matrix (matrix.h)}

\section{Introduction}

Functions to manage matrices are defined in \texttt{matrix.h} file.\\

\section{Data structures}

Some data structures are defined in \BI \ to manage matrices.

\subsection{\texttt{biaMatrix} data structure} \label{sec:biaMatrix}

This data structure is used to store a matrix. \textbf{biaMatrix} data structure is defined in figure \ref{fig:biaMatrix} where:

\begin{description}
\item[intRows] number of rows.
\item[intCols] number of columns.
\item[dblCoefs] pointer to store matrix coeficients.
\end{description}

\begin{figure}[!h]
\begin{verbatim}
typedef struct {
  int intRows,
      intCols;

  double **dblCoefs;
  } biaMatrix;   
\end{verbatim}
\caption{biaMatrix data structure.} \label{fig:biaMatrix}
\end{figure}

\FloatBarrier

\section{Matrix creation}

\BI \ includes functions to create some kind of matrices.

\subsection{\texttt{identityMatrix} function}

This function stores the identity matrix with order taken from intRows member of \texttt{ptMatrix}:
%
\begin{displaymath}
  \left( \begin{array}{ccccc}
    1 & 0 & 0 & 0 & 0 \\
    0 & 1 & \ddots & 0 & 0 \\
    0 & \ddots & \ddots & \ddots & 0 \\
    0 & 0 & \ddots & 1 & 0 \\
    0 & 0 & 0 & 0 & 1 \\
  \end{array} \right)
\end{displaymath}
%
The definition of this function:
%
\begin{verbatim}
void identityMatrix(biaMatrix *ptMatrix);  
\end{verbatim}
%
where:
%
\begin{description}
\item[*ptMatrix] matrix that has to be created before using this function. Memory allocation for \texttt{dblCoefs} must be done before using this function.
\end{description}
%
\ \\
%
\note{\texttt{intRows} is used to get the matrix order.}

\subsection{\texttt{scalingMatrix} function}

This function stores the scaling matrix with factor $\lambda$ and order taken from intRows member of \texttt{ptMatrix}:
%
\begin{displaymath}
  \left( \begin{array}{ccccc}
    \lambda & 0 & 0 & 0 & 0 \\
    0 & \lambda & \ddots & 0 & 0 \\
    0 & \ddots & \ddots & \ddots & 0 \\
    0 & 0 & \ddots & \lambda & 0 \\
    0 & 0 & 0 & 0 & \lambda \\
  \end{array} \right)
\end{displaymath}
%
The definition of this function:
%
\begin{verbatim}
void scalingMatrix(biaMatrix *ptMatrix, double dblFactor);  
\end{verbatim}
%
where:
%
\begin{description}
\item[*ptMatrix] matrix that has to be created before using this function. Memory allocation for \texttt{dblCoefs} must be done before using this function.
\end{description}
%
\ \\
%
\note{\texttt{intRows} is used to get the matrix order.}

\subsection{\texttt{nullMatrix} function}

This function stores the null matrix with order taken from intRows member of \texttt{ptMatrix}:
%
\begin{displaymath}
  \left( \begin{array}{ccccc}
    0 & 0 & 0 & 0 & 0 \\
    0 & 0 & \ddots & 0 & 0 \\
    0 & \ddots & \ddots & \ddots & 0 \\
    0 & 0 & \ddots & 0 & 0 \\
    0 & 0 & 0 & 0 & 0 \\
  \end{array} \right)
\end{displaymath}
%
The definition of this function:
%
\begin{verbatim}
void nullMatrix(biaMatrix *ptMatrix);  
\end{verbatim}
%
where:
%
\begin{description}
\item[*ptMatrix] matrix that has to be created before using this function. Memory allocation for \texttt{dblCoefs} must be done before using this function.
\end{description}
%
\ \\
%
\note{\texttt{intRows} and \texttt{intCols} is used to get the matrix order.}

\section{Matrix operations}

\subsection{\texttt{transposeMatrix} function}

This function stores the transpose matrix of a given matrix.\\ \\
%
The definition of this function:
%
\begin{verbatim}
void transposeMatrix(biaMatrix *ptMatrix, biaMatrix *ptRes);  
\end{verbatim}
%
where:
%
\begin{description}
\item[*ptMatrix] matrix to get its transpose matrix.
\item[*ptRes] matrix to store the transpose matrix. Memory has to be preallocated before using this function.
\end{description}
%
\ \\
%
\note{\texttt{intRows} and \texttt{intCols} is used to get the matrix order.}

\section{Matrix checks}

\subsection{\texttt{isIdentityMatrix} function}

This function checks if a matrix is the identity matrix.\\ \\
%
The definition of this function:
%
\begin{verbatim}
int isIdentityMatrix(biaMatrix *ptMatrix);
\end{verbatim}
%
where:
%
\begin{description}
\item[*ptMatrix] matrix to check.
\end{description}
%

\subsection{\texttt{isNullMatrix} function}

This function checks if a matrix is a null matrix.\\ \\
%
The definition of this function:
%
\begin{verbatim}
int isNullMatrix(biaMatrix *ptMatrix, double dblTol);
\end{verbatim}
%
where:
%
\begin{description}
\item[*ptMatrix] matrix to check.
\item[dblTol] if a matrix element is minor than this value it is assumed it is a null element.
\end{description}
%

\subsection{\texttt{isSymmetricMatrix} function}

This function checks if a matrix is a symmetric matrix.\\ \\
%
The definition of this function:
%
\begin{verbatim}
int isSymmetricMatrix(biaMatrix *ptMatrix);
\end{verbatim}
%
where:
%
\begin{description}
\item[*ptMatrix] matrix to check.
\end{description}
%


%
% roots.h
%

%
% roots.h
%

\chapter{Roots approximation (roots.h)} \label{sec:roots}

\section{Introduction}

Functions to compute function's roots approximation are defined in \texttt{roots.h} file.

\section{Data structures}

Some data structures are defined in \BI to manage roots.

\subsection{\texttt{biaRealRoot} data structure} \label{sec:biaRealRoot}

This data structure is used to store data for root approximation.\\

Data structure is defined in figure \ref{fig:biaRealRoot} where:
%
\begin{description}
\item[intNMI] maximum number of iterations to get the root with a maximum error of \texttt{dblTol}.
\item[intIte] iterations used to get the root.
\item[dblx0] initial approximation to get the root.
\item[dblRoot] root approximation.
\item[dblTol] maximum tolerance when calculating the root.
\item[dblError] error in root approximation. Difference between the las two root approximations.
\end{description}
%
\begin{figure}[!h]
\begin{verbatim}
typedef struct {
  int intNMI,
      intIte;

  double dblx0,
         dblRoot,
         dblTol,
         dblError;
  } biaRealRoot;
\end{verbatim}
\caption{biaRealRoot data structure.} \label{fig:biaRealRoot}
\end{figure}

\FloatBarrier

\section{Function roots approximation}

\subsection{\texttt{newtonPol} function}

This function approaches a polynomial root using the \textbf{Newton} method.\\ \\
%
The definition of this function:
%
\begin{verbatim}
int newtonPol(biaPol *ptPol, biaRealRoot *ptRoot);  
\end{verbatim}
%
The following codes are returned:
%
\begin{center}
\begin{tabular}{|l|l|}
\hline
\textbf{BIA\_MEM\_ALLOC} & Memory allocation error \\
\hline
\textbf{BIA\_ZERO\_DIV} & Division by zero \\
\hline
\textbf{BIA\_TRUE} & the root was computed satisfying the problem \\
                   & conditions (\texttt{intMNI} and \texttt{dblTol}); \\
\hline
\textbf{BIA\_FALSE} & Root approximation could not be calculated \\
                    & satisfying the requirements (\texttt{intMNI} and \texttt{dblTol}). \\
\hline
\end{tabular}
\end{center}
%
The following values in \texttt{*ptRoot} need to be initialized:
%
\begin{description}
\item[intMNI] maximun number of iterations to compute the root.
\item[dblx0] initial approximation.
\item[dblTol] tolerance to compute de root.
\end{description}
%
The following data will be stored:
%
\begin{description}
\item[intIte] iterations used to compute the root.
\item[dblRoot] approximation of the root.
\item[dblError] error in the approximation.
\end{description}
%
\note{When two consecutive approximations are close enough, \texttt{dblTol}, last approximation will be considered as good and will be stored in \texttt{*biaRealRoot *ptRoot} in \texttt{dblRoot}.}
%

\subsection{\texttt{newtonMethod} function}

This function approaches a function's root using the \textbf{Newton} method.\\ \\
%
The definition of this function:
%
\begin{verbatim}
int newtonMethod(biaRealRoot *ptRoot, 
       int (*func)(double dblx0, double *ptRes),
       int (*der)(double dblx0, double *ptRes));  
\end{verbatim}
%
Function's pointers are used to avoid having to recode the C function every time a root function need to be approximated for different mathematical functions.\\ \\
%
The following codes are returned:
%
\begin{center}
\begin{tabular}{|l|l|}
\hline
\textbf{BIA\_ZERO\_DIV} & Division by zero \\
\hline
\textbf{BIA\_TRUE} & the root was computed satisfying the problem \\
                   & conditions (\texttt{intMNI} and \texttt{dblTol}); \\
\hline
\textbf{BIA\_FALSE} & Root approximation could not be calculated \\
                    & satisfying the requirements (\texttt{intMNI} and \texttt{dblTol}). \\
\hline
\end{tabular}
\end{center}
%
where:
%
\begin{description}
\item[*ptRoot] is a pointer to a \texttt{biaRealRoot} variable.
\item[*func] pointer to a function implementing the function which root is going to be computed.
\item[*der] pointer to a function implementing the derivative of the function which root is going to be computed.
\end{description}
%
The following values in \texttt{*ptRoot} need to be initialized:
%
\begin{description}
\item[intMNI] maximun number of iterations to compute the root.
\item[dblx0] initial approximation.
\item[dblTol] tolerance to compute de root.
\end{description}
%
The following data is stored:
%
\begin{description}
\item[intIte] iterations used to compute the root.
\item[dblRoot] approximation of the root.
\item[dblError] error in the approximation.
\end{description}
%
\note{When two consecutive approximations are close enough, \texttt{dblTol}, last approximation will be considered as good and will be stored in \texttt{*biaRealRoot *ptRoot} in \texttt{dblRoot}.}

\subsubsection{Usage example}

To approximate a root for the function $f(x) = \sqrt{2}$ to C functions need to be created. A C function for the mathematical function implementation:
%
\begin{verbatim}
/* f(x) = x^2 - 2 */
int myfunc(double x0, double *fx0) {
  *fx0 = (double)(x0 * x0 - 2.);
  return BIA_TRUE;
}  
\end{verbatim}
%
A C function for the mathematical derivative function implementation:
%
\begin{verbatim}
/* f'(x) = 2*x */
int myfuncder(double x0, double *fx0) {
  *fx0 = 2.*x0;
  return BIA_TRUE;
}  
\end{verbatim}
%
Both functions must meet the following requirements:
%
\begin{itemize}
\item An integer value is returned, \textbf{BIA\_TRUE} when function is evaluated in $x0$ and \textbf{BIA\_ZERO\_DIV} if a division by zero takes place.
\item $x0$ value to evaluate the function.
\item $*fx0$ pointer to a double to store the function's value in $x0$.
\end{itemize}
%
So to approximate function's root using Newton Method:
%
\begin{verbatim}
i = newtonMethod(&myRoot, &myfunc, &myfuncder);  
\end{verbatim}


%
% rngkutta.h
%

%
% RNGKUTTA.H
%

\chapter{Runge-Kutta methods (rngkutta.h)}

\section{Introduction}

\textbf{Runge-Kutta} are a family of implicit and explicit iterative methods used to approximate solutions of ordinary differential equations or \textbf{ODE}.\\

Butcher matricial notation is used in this implementation.\\

\section{Data structures}

\subsection{\texttt{biaButcherArray} data structure} \label{sec:biaButcherArray}

This structure is used to store the Butcher matricial notation.\\

Data structure is defined in figure \ref{fig:biaButcherArray} where:
%
\begin{description}
%
\item[intStages] method stages.
%
\item[*dblC] $c_i$ coefficients stored in an array with size \texttt{intStages}.
%
\item[*dblB] $b_i$ coefficients stored in an array with size \texttt{intStages}.
%
\item[**dblMatrix] matrix to store $a_{i,j}$ method's coeficients.  
%
\end{description}

\begin{figure}[!h]
\begin{verbatim}
typedef struct {
  double  *dblC,
          *dblB,
          **dblMatrix;

  int     intStages;
} biaButcherArray;
\end{verbatim}
\caption{biaButcherArray data structure.} \label{fig:biaButcherArray}
\end{figure}
%
\FloatBarrier

\note{See appendix \ref{ch:runge} if you need information about Butcher matricial notation.}

\subsection{\texttt{biaDataRK} data structure} \label{sec:biaDataRK}

This structure is used to store all the data needed to apply a Runge-Kutta method.\\

Data structure is defined in figure \ref{fig:biaDataRK} where:
%
\begin{description}
%
\item[intNumApprox] number of approximations to be done (size of the array \texttt{dblPoints}).
%
\item[intImplicit] when the Runge-Kutta method is implicit or not. The following constants are defined in the header file:
%
\begin{center}
\begin{tabular}{|c|c|}
\hline
\textbf{Name} & \textbf{Value} \\
\hline
\textbf{BIA\_IMPLICIT\_RK\_TRUE} & $0$ \\
\hline
\textbf{BIA\_IMPLICIT\_RK\_FALSE} & $1$ \\
\hline
\end{tabular}
\end{center}  
%
\item[*dblPoints] array with dimension \texttt{intNumApprox} and its elements will be the approximations in $x_i$ where:
%
\begin{displaymath}
x_i = dblFirst + i \cdot dblStepSize \qquad \textrm{where} \qquad 0 \le i < intNumApprox     
\end{displaymath}
%
\item[dblStepSize] method's step-size.
%
\item[dblFirst] first point used to compute all the approximations. The value of the function in this point is known (initial condition).
%
\item[dblLast] last point in which approximations will be computed.
%
\item[strCoefs] variable of type biaButcherArray (section \ref{sec:biaButcherArray}) storing Butcher matricial notation.
%
\end{description}

\begin{figure}[!h]
\begin{verbatim}
typedef struct {
  int intNumApprox,
      intImplicit;

  double  *dblPoints,
          dblStepSize,
          dblFirst,
          dblLast;

  biaButcherArray strCoefs;
} biaDataRK;
\end{verbatim}
\caption{biaDataRK data structure.} \label{fig:biaDataRK}
\end{figure}
%
\FloatBarrier

\section{Node number calculations}

\subsection{\texttt{intNodeNumber} function}

This function gets the number of nodes that can be placed in an interval. All nodes are equidistant.\\

The definition of this function:
%
\begin{verbatim}
int intNodeNumber(double dblLong, double dblStepSize)
\end{verbatim}
%
where:
%
\begin{description}
\item[dblLong] interval length.
\item[dblStepSize] distance between two nodes.
\end{description}

The function returns the number of nodes that can be placed.\\ \\
%
\note{Arguments are supposed to be different from zero.}
\ \\
\note{Arguments are supposed to be positive.}

\section{Explicit Runge-Kutta methods (scalar problems)}

\subsection{\texttt{ExplicitRungeKutta} function}

This function solves an \emph{I.V.P.} using an explicit Runge-Kutta method.\\

The definition of this function:
%
\begin{verbatim}
int explicitRungeKutta(biaDataRK *ptData, 
               double (*IVP)(double dblX, double dblY)
\end{verbatim} 
%
where:
%
\begin{description}
\item[ptData] pointer to a \textbf{biaDataRK} variable\footnote{Section (\ref{sec:biaDataRK}).}. This variable contains all the necessary data to solve the \emph{I.V.P.}
\item[PVI] C function's pointer to a function implementing the \emph{O.D.E.}. This functions needs to have two double arguments and returns the value of the \emph{I.V.P.}:
\begin{description}
	\item[dblX] point where we want to evaluate the \emph{O.D.E.}
	\item[dblY] \emph{O.D.E.} value in \texttt{dblX} ($y_i \approx y(x_i)$).
\end{description}
\end{description}

The following codes are returned:
%
\begin{center}
\begin{tabular}{|l|l|}
\hline
\textbf{BIA\_MEM\_ALLOC} & Memory error allocation \\
\hline
\textbf{BIA\_TRUE} & Success \\
\hline
\end{tabular}
\end{center}

\subsubsection{Usage example}

For instance, to solve this \emph{I.V.P.}:
%
\begin{displaymath}
\left \{ \begin{array}{l}
y' = y(x) * \frac{x-y(x)}{x^2} \\
y(1) = 2
\end{array} \right.
\end{displaymath}
%
the implementation of the \texttt{IVP} would be:
%
\begin{verbatim}
double IVP(double dblX, double dblY) {
  double  dblResultado;

  dblRes = dblY*((dblX-dblY)/(dblX*dblX));

  return (dblRes);
}
\end{verbatim}




\begin{center}
\emph{intResultado = \textbf{ExplicitRungeKutta}(\&varstrDatRK, PVI);}
\end{center}

Resolver\'{\i}a el \emph{P.V.I.} representado por la funci\'on \emph{PVI} 
utilizando los datos almacenados en la variable, del tipo \emph{DatosRK},
\emph{varstrDatRK} y almacenar\'{\i}a en \emph{intResultado} el c\'odigo
devuelto por la funci\'on.

\subsection{\texttt{RungeKuttaClasico} function}

Funci\'on que inicializa los coeficientes para el m\'etodo \emph{Runge-Kutta 
Cl\'asico}, el cual es un m\'etodo de $4$ etapas y orden $4$.\newline

La notaci\'on matricial del m\'etodo es la siguiente:

\begin{center}
$
\begin{array}{c|cccc}
0 & 0 \\
\frac{1}{2} & \frac{1}{2} & 0 \\
\frac{1}{2} & 0 & \frac{1}{2} & 0 \\
1 & 0 & 0 & 1 & 0 \\
\hline
 & \frac{1}{6} & \frac{1}{3} & \frac{1}{3} & \frac{1}{6} \\
\end{array}
$
\end{center}

El prototipo de esta funci\'on es el siguiente:

\begin{center}
\emph{int \textbf{RungeKuttaClasico}(DatosRK *ptstrDatos)}
\end{center}

\begin{description}
\item[ptstrDatos] puntero a una variable de \emph{estructura} del tipo
\emph{DatosRK}.
\end{description}

La funci\'on devuelve los siguientes c\'odigos:

\begin{center}
\begin{tabular}{|l|l|}
\hline
\textbf{ERR\_AMEM} & Hubo un error en la asignaci\'on de memoria. \\
\hline
\textbf{TRUE} & Se inicializaron con \'exito los coeficientes. \\
\hline
\end{tabular}
\end{center}

Por ejemplo:

\begin{center}
\emph{intResultado = \textbf{RungeKuttaClasico}(\&varstrDatRK);}
\end{center}

Inicializar\'{\i}a los coeficientes del m\'etodo en la variable 
\emph{varstrDatRK}, en \emph{intResultado} el valor \textbf{TRUE} si se pudieron
inicializar los coeficientes y en caso contrario \textbf{ERR\_AMEM}.

\subsection{MetodoHeun}
Funci\'on que inicializa los coeficientes para el m\'etodo de \emph{Heun}, el
cual es un m\'etodo \emph{Runge-Kutta} de $3$ etapas y orden $3$.\newline

La notaci\'on matricial del m\'etodo es la siguiente:

\begin{center}
$
\begin{array}{c|ccc}
0 & 0 \\
\frac{1}{3} & \frac{1}{3} & 0 \\
\frac{2}{3} & 0 & \frac{2}{3} & 0 \\
\hline
 & \frac{1}{4} & 0 & \frac{3}{4}
\end{array}
$
\end{center}

El prototipo de esta funci\'on es el siguiente:

\begin{center}
\emph{int \textbf{MetodoHeun}(DatosRK *ptstrDatos)}
\end{center}

\begin{description}
\item[ptstrDatos] puntero a una variable de \emph{estructura} del tipo
\emph{DatosRK}.
\end{description}

La funci\'on devuelve los siguientes c\'odigos:

\begin{center}
\begin{tabular}{|l|l|}
\hline
\textbf{ERR\_AMEM} & Hubo un error en la asignaci\'on de memoria. \\
\hline
\textbf{TRUE} & Se inicializaron con \'exito los coeficientes. \\
\hline
\end{tabular}
\end{center}

Por ejemplo:

\begin{center}
\emph{intResultados = \textbf{MetodoHeun}(\&varstrDatRK);}
\end{center}

Inicializar\'{\i}a los coeficientes del m\'etodo en la variable
\emph{varstrDatRK}, en \emph{intResultado} el valor \textbf{TRUE} si se pudieron
inicializar los coeficientes y en caso contrario \textbf{ERR\_AMEM}.

\subsection{MetodoKutta}

Funci\'on que inicializa los coeficientes para el m\'etodo de \emph{Kutta}, el
cual es un m\'etodo \emph{Runge-Kutta} de $3$ etapas y orden $3$.\newline

La notaci\'on matricial del m\'etodo es la siguiente:

\begin{center}
$
\begin{array}{c|ccc}
0 & 0 \\
\frac{1}{2} & \frac{1}{2} & 0 \\
1 & -1 & 2 & 0 \\
\hline
 & \frac{1}{6} & \frac{2}{3} & \frac{1}{6}
\end{array}
$
\end{center}

El prototipo de esta funci\'on es el siguiente:

\begin{center}
\emph{int \textbf{MetodoKutta}(DatosRK *ptstrDatos)}
\end{center}

\begin{description}
\item[ptstrDatos] puntero a una variable de \emph{estructura} del tipo
\emph{DatosRK}.
\end{description}

La funci\'on devuelve los siguientes c\'odigos:

\begin{center}
\begin{tabular}{|l|l|}
\hline
\textbf{ERR\_AMEM} & Hubo un error en la asignaci\'on de memoria. \\
\hline
\textbf{TRUE} & Se inicializaron con \'exito los coeficientes. \\
\hline
\end{tabular}
\end{center}

Por ejemplo:

\begin{center}
\emph{intResultado = \textbf{MetodoKutta}(\&varstrDatRK);}
\end{center}

Inicializar\'{\i}a los coeficientes del m\'etodo en la variable
\emph{varstrDatRK}, en \emph{intResultado} el valor \textbf{TRUE} si se pudieron
inicializar los coeficientes y en caso contrario \textbf{ERR\_AMEM}.

\subsection{\texttt{EulerModificado} function}

Funci\'on que inicializa los coeficientes para el m\'etodo de \emph{Euler 
modificado}, el cual es un m\'etodo \emph{Runge-Kutta} de $2$ etapas y 
orden $2$.\newline

La notaci\'on matricial del m\'etodo es la siguiente:

\begin{center}
$
\begin{array}{c|cc}
0 & 0 \\
\frac{1}{2} & \frac{1}{2} & 0 \\
\hline
 & 0 & 1
\end{array}
$
\end{center}

El prototipo de esta funci\'on es el siguiente:

\begin{center}
\emph{int \textbf{EulerModificado}(DatosRK *ptstrDatos)}
\end{center}

\begin{description}
\item[ptstrDatos] puntero a una variable de \emph{estructura} del tipo
\emph{DatosRK}.
\end{description}

La funci\'on devuelve los siguientes c\'odigos:

\begin{center}
\begin{tabular}{|l|l|}
\hline
\textbf{ERR\_AMEM} & Hubo un error en la asignaci\'on de memoria. \\
\hline
\textbf{TRUE} & Se inicializaron con \'exito los coeficientes. \\
\hline
\end{tabular}
\end{center}

Por ejemplo:

\begin{center}
\emph{intResultado = \textbf{EulerModificado}(\&varstrDatRK);}
\end{center}


Inicializar\'{\i}a los coeficientes del m\'etodo en la variable
\emph{varstrDatRK}, en \emph{intResultado} el valor \textbf{TRUE} si se pudieron
inicializar los coeficientes y en caso contrario \textbf{ERR\_AMEM}.

\subsection{\texttt{EulerMejorado} function}

Funci\'on que inicializa los coeficientes para el m\'etodo de \emph{Euler mejorado},
el cual es un m\'etodo \emph{Runge-Kutta} de $2$ etapas y orden $2$.\newline

La notaci\'on matricial del m\'etodo es la siguiente:

\begin{center}
$
\begin{array}{c|cc}
0 & 0 \\
1 & 1 & 0 \\
\hline
 & \frac{1}{2} & \frac{1}{2}
\end{array}
$
\end{center}

El prototipo de esta funci\'on es el siguiente:

\begin{center}
\emph{int \textbf{EulerMejorado}(DatosRK *ptstrDatos)}
\end{center}

\begin{description}
\item[ptstrDatos] puntero a una variable de \emph{estructura} del tipo
\emph{DatosRK}.
\end{description}

La funci\'on devuelve los siguientes c\'odigos:

\begin{center}
\begin{tabular}{|l|l|}
\hline
\textbf{ERR\_AMEM} & Hubo un error en la asignaci\'on de memoria. \\
\hline
\textbf{TRUE} & Se inicializaron con \'exito los coeficientes. \\
\hline
\end{tabular}
\end{center}

Por ejemplo:

\begin{center}
\emph{intResultado = \textbf{EulerMejorado}(\&varstrDatRK);}
\end{center}


Inicializar\'{\i}a los coeficientes del m\'etodo en la variable
\emph{varstrDatRK}, en \emph{intResultado} el valor \textbf{TRUE} si se pudieron
inicializar los coeficientes y en caso contrario \textbf{ERR\_AMEM}.





Todas estas funciones suponen que la variable de \emph{estructura}, del tipo
\emph{DatosRK}\footnote{Apartado (\ref{sec:datosRK}) en la p\'agina 
\pageref{sec:datosRK}}, no tienen dimensionados los punteros en ella 
contenidos, raz\'on por la cual ser\'a necesario liberar la memoria asignada
a estos antes de pasarle como parametro una variable de este tipo a una de
las siguientes funciones(siempre y cuando se hayan dimensionado dichos
punteros).\newline

Hay que destacar que \textbf{NO} se inicializan todos los miembros de esta
estructura, s\'olo aquellos miembros que contienen los coeficientes del 
m\'etodo.\newline

Los siguientes miembros \textbf{NO} se inicializan:
%
\begin{description}
\item[intNumAprox]
\item[dblPuntos]
\item[dblPaso]
\item[dblInicio]
\item[dblFinal]
\end{description}

Estos miembros son independientes del m\'etodo, dependen del problema que
se quiera resolver y tendr\'an que ser inicializados por el usuario.


%
% pi.h
%

%
% PI.H
%

\chapter{$\pi$ computation (pi.h)}

\section{Introduction}

In this chapter several methods will be shown to compute $\pi$ digits.\\

The following libraries will be used:
%
\begin{itemize}
\item \href{http://www.openmp.org/}{OpenMP} the Open Multi-Processing API which provides multiplatform shared memory capabilities for parallel programming.
\item \href{https://gmplib.org/}{GMP} the GNU Multiple Precision Arithmetic Library.
\end{itemize}

\section{The midpoint rule method}

To compute $\pi$ value using the midpoint rule the following is used:
%
\begin{displaymath}
\int_0^1 \frac{1}{1+x^2} dx = \frac{\pi}{4}
\end{displaymath}
%
So:
%
\begin{displaymath}
\pi = 4 \cdot \int_0^1 \frac{1}{1+x^2} dx
\end{displaymath}

\subsection{\texttt{threadedPiMidPointRule} function}

This function approaches $\pi$ using the midpoint rule.\\

The definition of this function:
%
\begin{verbatim}
long double threadedPiMidPointRule(int intThreads, int intN);  
\end{verbatim}
%
where:
%
\begin{description}
\item[intThreads] number of threads to be used.
\item[intN] number of subintervals to be used.
\end{description}

\note{To compute the integral the midpoint rule will be used. The implementation used for the midpoint rule used is the one provided by \BI \ in section \ref{sec:threadedMidPointRule}.}

\section{Chudnovsky algorithm}

The Chudnovsky algorithm uses the following to compute $\pi$:
%
\begin{displaymath}
\pi \cdot \sum_{i=0}^{\infty} \frac{(6\cdot i)!\cdot(13591409 + 545140134 \cdot i)}{(3\cdot i)!\cdot(i!)^3\cdot (-640320)^{3\cdot i}}  = 426880 \cdot \sqrt{10005}
\end{displaymath}
%
For high performance computation:
%
\begin{displaymath}
\pi \cdot \sum_{i=0}^{\infty} \frac{(6\cdot i)!\cdot(13591409 + 545140134 \cdot i)}{(3\cdot i)!\cdot(i!)^3\cdot (-262537412640768000)^{i}}  = 426880 \cdot \sqrt{10005}
\end{displaymath}
%

\subsection{\texttt{chudnovskyPi} function}


%%%%%%%%%%%%%
% Appendix  %
%%%%%%%%%%%%%

\appendix

%
% Runge - Kutta 
%

%
% Apendice sobre metodos Runge - Kutta 
%

\chapter{Runge-Kutta methods} \label{sec:Runge}

This appendix is intended to help to know how Runge-Kutta methods are implemented and used in this library.

\section{What is a Runge-Kutta method?}

\textbf{Runge-Kutta} methods are a family of numerical methods to approach solutions of ordinary differential equations (O.D.E). These methods are iterative methods used to solve ``\emph{initial problem value}'' (\textbf{I.P.V}) or ``\emph{Cauchy problem}''.\\

These methods are only-one-step methods with a fixed size for the method step\footnote{It is also possible to implement methods with a variable step known as \emph{embedding}.}.\\

\subsection{What is a I.V.P.?}

An \emph{I.V.P.} is:

\begin{equation} \label{eq:IVP}
\left\{ \begin{array}{l}
y' = f(x, y(x))\\
y(x_0) = y_0\\
\end{array} \right.
\end{equation}
%
So $y'$ is a function depending on the variable $x$, and the function $y(x)$. $y(x)$ is the solution of the equation \ref{eq:IVP} and the point $(x_0,y_0)$ belongs to the curve $y(x)$.\\

Solving the \emph{I.V.P.} \ref{eq:IVP} is finding a function $y(x)$ such as the equation \ref{eq:IVP} is met.\\

An example of a \emph{I.V.P.}:
%
\begin{equation} \label{eq:IVPej}
\left\{ \begin{array}{l}
y' = \frac{x * y(x) - y(x)^2}{x^2} \\
y(1) = 2 \\
\end{array} \right.
\end{equation}
%
The solution of the \ref{eq:IVPej} will be:
%
\begin{equation}
y(x) = \frac{x}{\frac{1}{2}+\ln x}
\end{equation}

\section{Runge-Kutta's method notation}

$y(x_i)$ will be the exact value of the function $y(x)$ evaluated in $x_i$.\\ \\
$y_i$ will be the approximation of the function $y(x)$ in the point $x_i$.\\ \\
$h$ is the step used by the method in each iteration.

\subsection{General formulation}

A $s$-stages \textbf{Runge-Kutta}'s method formulation is:
%
\begin{equation}
y_{n+1} = y_{n} + h \cdot \sum_{i=0}^{s-1} b_i \cdot k_i
\end{equation}
%
where:
%
\begin{equation}
k_i = f(x_n + c_i \cdot h, y_n + h \cdot \sum_{j=0}^{s-1} a_{i,j} \cdot k_j)
\end{equation}
%
satisfying:
%
\begin{equation}
\sum_{j=0}^{s-1} a_{i,j} = c_i
\end{equation}

\subsection{Matricial notation (Butcher's)}

Matricial notation is used to represent method's coeficients using a matrix.\\

For a $s$-stages \textbf{Runge-Kutta} method the matricial notation will be:
%
\begin{center}
\begin{displaymath}
\begin{array}{c|ccc}
c_0 & a_{0,0} & \cdots \cdots & a_{0,s-1} \\
\vdots & \vdots & & \vdots \\
\vdots & \vdots & & \vdots \\
c_{s-1} & a_{s-1,0} & \cdots \cdots & a_{s-1,s-1} \\
\hline
 & b_0 & \cdots \cdots & b_{s-1} \\
\end{array}
\end{displaymath}
\end{center}

\note{In section \ref{sec:biaButcherArray} is shown a data structure used to store the Butcher array.}

\section{Runge-Kutta types}

There are several types of \textbf{Runge-Kutta} methods.

\subsection{Implicit Runge-Kutta}

A \textbf{Runge-Kutta} method is said to be implicit when the $a_{i,j} \neq 0$ for some $j > i$.\\

The $2$-stages Gauss method is an implicit \textbf{Runge-Kutta} method of $2$-stages:
%
\begin{center}
\begin{displaymath}
\begin{array}{c|cc}
\frac{3-\sqrt 3}{6} & \frac{1}{4} & \frac{3-2*\sqrt 3}{12} \\
\frac{3+\sqrt 3}{6} & \frac{3+2*\sqrt 3}{12} &\frac{1}{4} \\
\hline
 & \frac{1}{2} & \frac{1}{2}
\end{array}
\end{displaymath}
\end{center}

\subsection{Semi-implicit Runge-Kutta}

A \textbf{Runge-Kutta} method is said to be semi-implicit when the $a_{i,j} = 0$ when $j > i$.\\

A $2$-stages semi-implicit \textbf{Runge-Kutta} method:
%
\begin{center}
\begin{displaymath}
\begin{array}{c|cc}
\frac{3+\sqrt 3}{6} & \frac{3+\sqrt 3}{6} & 0 \\
\frac{3-\sqrt 3}{6} & \frac{-\sqrt 3}{3} & \frac {3+\sqrt 3}{6} \\
\hline
 & \frac{1}{2} & \frac{1}{2}
\end{array}
\end{displaymath}
\end{center}

\subsection{Explicit Runge-Kutta}

A \textbf{Runge-Kutta} method is said to be explicit when the $a_{i,j} = 0$ when $j \geq i$.\\

A $4$-stages explicit \textbf{Runge-Kutta} method also known as ``\textbf{classic Runge-Kutta}'':
%
\begin{center}
\begin{displaymath}
\begin{array}{c|cccc}
0 & 0 \\
\frac{1}{2} & \frac{1}{2} & 0 \\
\frac{1}{2} & 0 & \frac{1}{2} & 0 \\
1 & 0 & 0 & 1 & 0 \\
\hline
 & \frac{1}{6} & \frac{1}{3} & \frac{1}{3} & \frac{1}{6}
\end{array}
\end{displaymath}
\end{center}


\end{document}
