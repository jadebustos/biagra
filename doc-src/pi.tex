%
% PI.H
%

\chapter{$\pi$ computation (pi.h)}

\section{Introduction}

In this chapter several methods will be shown to compute $\pi$ digits.\\

The following libraries will be used:
%
\begin{itemize}
\item \href{http://www.openmp.org/}{OpenMP} the Open Multi-Processing API which provides multiplatform shared memory capabilities for parallel programming.
\item \href{https://gmplib.org/}{GMP} the GNU Multiple Precision Arithmetic Library.
\end{itemize}

\section{The midpoint rule method}

To compute $\pi$ value using the midpoint rule the following is used:
%
\begin{displaymath}
\int_0^1 \frac{1}{1+x^2} dx = \frac{\pi}{4}
\end{displaymath}
%
So:
%
\begin{displaymath}
\pi = 4 \cdot \int_0^1 \frac{1}{1+x^2} dx
\end{displaymath}

\subsection{\texttt{threadedPiMidPointRule} function}

This function approaches $\pi$ using the midpoint rule.\\

The definition of this function:
%
\begin{verbatim}
long double threadedPiMidPointRule(int intThreads, int intN);  
\end{verbatim}
%
where:
%
\begin{description}
\item[intThreads] number of threads to be used.
\item[intN] number of subintervals to be used.
\end{description}

\note{To compute the integral the midpoint rule will be used. The implementation used for the midpoint rule used is the one provided by \BI \ in section \ref{sec:threadedMidPointRule}.}

\section{Chudnovsky algorithm}

The Chudnovsky algorithm uses the following to compute $\pi$:
%
\begin{displaymath}
\pi \cdot \sum_{i=0}^{\infty} \frac{(6\cdot i)!\cdot(13591409 + 545140134 \cdot i)}{(3\cdot i)!\cdot(i!)^3\cdot (-640320)^{3\cdot i}}  = 426880 \cdot \sqrt{10005}
\end{displaymath}
%
For high performance computation:
%
\begin{displaymath}
\pi \cdot \sum_{i=0}^{\infty} \frac{(6\cdot i)!\cdot(13591409 + 545140134 \cdot i)}{(3\cdot i)!\cdot(i!)^3\cdot (-262537412640768000)^{i}}  = 426880 \cdot \sqrt{10005}
\end{displaymath}
%

\subsection{\texttt{chudnovskyPi} function}

This function approaches $\pi$ using the Chudnovsky algorithm.\\

The definition of this function:
%
\begin{verbatim}
int chudnovskyPi(int intThreads, int intPiDigits, char *ptPi);  
\end{verbatim}
%
where:
%
\begin{description}
\item[intThreads] number of threads to be used.
\item[intPiDigits] number of $\pi$ digits to compute.
\item[ptPi] char point to store $\pi$ as a string.
\end{description}

\note{This algorithm uses the \textbf{GNU Multiple Precision Library (GMP)} to be able to perform all the computations. C default number types do not provide enough precision.}

\note{This algorithm uses \textbf{OpenMP} to be able to parallelize computations.}

\note{\texttt{ptPi} must be initialized to store $\pi$ as a string. For instance to store $100$ $\pi$ digits the size to allocate memory is $102$ to include the ``\texttt{.}'' character and the ``null character'' at the end of the string.}
