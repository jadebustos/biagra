%
% PI.H
%

\chapter{$\pi$ computation (pi.h)}

\section{Introduction}

In this chapter several methods will be shown to compute $\pi$ digits.\\

The following libraries will be used:
%
\begin{itemize}
\item \href{http://www.openmp.org/}{OpenMP} the Open Multi-Processing API which provides multiplatform shared memory capabilities for parallel programming.
\item \href{https://gmplib.org/}{GMP} the GNU Multiple Precision Arithmetic Library.
\end{itemize}

\section{The mid point rule}

\begin{displaymath}
\pi = 4 \cdot \int_0^1 \frac{1}{1+x^2} dx \approx \frac{4}{n} \cdot \sum_{i=0}^n \frac{1}{1+x_i^2} \quad \textrm{where} \quad x_i = \frac{i -\frac{1}{2}}{n}
\end{displaymath}
