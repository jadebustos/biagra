%
% PI.H
%

\chapter{$\pi$ computation (pi.h)}

\section{Introduction}

In this chapter several methods will be shown to compute $\pi$ digits.\\

The following libraries will be used:
%
\begin{itemize}
\item \href{http://www.openmp.org/}{OpenMP} the Open Multi-Processing API which provides multiplatform shared memory capabilities for parallel programming.
\item \href{https://gmplib.org/}{GMP} the GNU Multiple Precision Arithmetic Library.
\end{itemize}

\section{The midpoint rule method}

To compute $\pi$ value using the midpoint rule the following is used:
%
\begin{displaymath}
\int_0^1 \frac{1}{1+x^2} dx = \frac{\pi}{4}
\end{displaymath}
%
So:
%
\begin{displaymath}
\pi = 4 \cdot \int_0^1 \frac{1}{1+x^2} dx
\end{displaymath}

\subsection{\texttt{threadedPiMidPointRule} function}

This function approaches $\pi$ using the midpoint rule.\\

The definition of this function:
%
\begin{verbatim}
long double threadedPiMidPointRule(int intThreads, int intN);  
\end{verbatim}
%
where:
%
\begin{description}
\item[intThreads] number of threads to be used.
\item[intN] number of subintervals to be used.
\end{description}

\note{To compute the integral the midpoint rule will be used. The implementation used for the midpoint rule used is the one provided by \BI \ in section \ref{sec:threadedMidPointRule}.}
